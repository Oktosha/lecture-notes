\documentclass[document.tex]{subfiles}

\begin{document}

\section{Введение. Сходимости векторов.}

\subsection{Введение.}

Математическая статистика -- это раздел теории вероятностей, который решает обратные задачи к классическим задачам в теории вероятностей.

Типичная задача в теории вероятностей -- это найти или оценить характеристики случайного эксперимента, зная его природу случайности.

Типичная задача в математической статистике -- по данным результатов случайного эксперимента выяснить природу его случайности.

\begin{example}[Классический пример]
	В городе есть $n$ жителей, $m$ из которых болеют. Считаем, что $n$ дано заранее.
	\begin{itemize}
		\item Задача ТВ: с какой вероятностью при известном $m$ в случайной выборке из $a$ жителей будет $b$ заболевших
		\item Задача МС: известно, что в выборке из $a$ жителей оказалось $b$ заболевших. Как в этом случае можно оценить $m$
	\end{itemize}
\end{example}

\begin{example}[Выборка]
	Предположим, что мы проводим эксперимент. Пусть дан какой-то физический прибор, и пусть $\xi$ -- случайная величина, описывающая результат измерения этим прибором, $\xi \sim P_{\xi}$ ($\xi$ имеет распределение $P_{\xi}$). Например, если прибор -- это счетчик Гейгера, то $\xi$ -- это уровень радиации, им зарегестрированный. Давайте также считать, что на время эксперимента распределение $\xi$ не меняется, и результат измерения прибора не зависит от предыдущих измерений.  Пусть $X_1, \dots, X_n$ -- эти результаты измерения в какие-то моменты времени. На языке теории вероятностей это можно переформулировать так: $X_1, \dots, X_n$ -- \textit{реализации независимых одинаково распределенных случайных величин} $\xi_1, \dots, \xi_n$. 

	Задача состоит в том, чтобы оценить $E\xi$ по этим самым $X_1, \dots, X_n$.
\end{example}

\begin{example}[Регрессионная модель]
	Пусть материальная точка движется по прямой, стартовав из точки $x_0$ с постоянной скоростью, равной $v_0$. Мы их не знаем, и будем считать, что это случайные величины. Пусть $x_1, \dots, x_n$ -- это измеренные нами положения этой материальной точки в моменты времени $t_1, \dots, t_n$ соответственно. Или, по другому, $x_1, \dots, x_n$ -- это реализации случайных величин $\xi_1, \dots, \xi_n$, причем $\xi_i$ отвечает за измеренный нами результат положения точки в момент времени $t_i$. Понятно, что эти случайные величины уже будут зависимы. Дополнительно положим, что \textit{погрешность измерения подчиняется нормальному распределению}. То есть: $\xi_i = x_0 + v_0 \cdot t_i + \varepsilon_i$, где $\varepsilon_i$ -- нормально распределенная случайная величина, отвечающая за ошибку $i$-того измерения.

	Задача заключается в том, чтобы оценить $x_0$ и $v_0$ по этим данным ($x_0, \dots, x_n, t_0, \dots, t_n$)
\end{example}

\begin{example}[Проверка на однородность]
	Пусть $X_1, \dots, X_n$ -- это результаты эксперимента в условиях $A$, а $Y_1, \dots, Y_m$ -- результаты того-же самого эксперимента в условиях $B$. Нужно выяснить, влияют ли эти условия на результат. (Если для сокращения записи отождествить результат эксперимента со случайной величиной, реализацией которой он является, а также считать, что $X_i \sim X$, $Y_i \sim Y$ то можно записать так: $X \stackrel{d}{\sim} Y$?)
\end{example}

\begin{remark}
	Как мы видим, задача матстатистики -- представить оптимальное решение на основе статистических данных. Типичная характерная черта задач -- это довольно большое количество дополнительных ограничений на природу явлений (независимость и одинаковая распределенность результата, нормальное распределение погрешностей и т.д.). Такие ограничения в реальных условиях иногда бывает трудно проверить, поэтому нужно быть крайне внимательным при применении какого-либо результата из матстатистики в реальных задачах. Однако такое требование к внимательности компенсируется тем, что результаты из матстатистики находят широкое применение в экспериментальной физике, машинном обучении, data mining и прочих облостях науки.
\end{remark}

\subsection{Сходимости случайных векторов}

\begin{definition}
	Пусть $\{\xi_n : n \in \mathbb{N}\}$ -- последовательность случайных векторов из $\mathbb{R}^m$. $\xi$ -- случайный вектор из $\mathbb{R}^m$. Говорят что:
	\begin{itemize}
		\item $\xi_n$ сходится к $\xi$ почти наверное (обозначение: $\xi_n \cae \xi$), если :
			$$P(\{\omega : \lim_{n \rightarrow \infty} \xi_n(\omega) = \xi(\omega)\}) = 1$$
		\item $\xi_n$ сходится к $\xi$ по вероятности (обозначение $\xi_n \cp \xi$), если:
			$$\forall \varepsilon > 0 : \lim_{n \rightarrow \infty} P(\{\omega : \|\xi_n(\omega) - \xi(\omega)\| > \varepsilon\}) = 0$$
		\item $\xi_n$ сходитiся к $\xi$ по распределению (обозначение $\xi_n \cd \xi$), если:
			$$\forall f \text{ -- ограниченной непрерывной функции $\mathbb{R}^m \mapsto \mathbb{R}$} : \lim_{n \rightarrow \infty}Ef(\xi_n) = Ef(\xi)$$
	\end{itemize}
\end{definition}

\begin{statement}
	Пусть $\xi_n = (\xi_n^1, \dots, \xi_n^m), \xi = (\xi^1, \dots, \xi^m)$. Тогда:
	\begin{itemize}
		\item $\xi_n \cae \xi \Leftrightarrow \forall j : \xi_n^j \cae \xi^j$
		\item $\xi_n \cp \xi \Leftrightarrow \forall j : \xi_n^j \cp \xi^j$
	\end{itemize}
\end{statement}

\begin{remark}
	Для сходимости по распределению это утверждение не верно
\end{remark}

\begin{definition}
	Функции распределения $F_{\xi_n}$ называются слабо сходящимися к $F_{\xi}$ (обозначение: $F_{\xi_n} \cw F_{\xi}$), если:
	$$\forall f \text{ -- ограниченной непрерывной функции $\mathbb{R} \mapsto \mathbb{R}$} : \int_{\Omega}f(x)dF_{\xi_n} \rightarrow \int_{\Omega}f(x)dF_{\xi}$$
\end{definition}

\begin{theorem}[Александрова]
	Пусть $\xi, \xi_1, \xi_2, \dots$ -- случайные величины. Тогда следующие утверждения эквивалентны:
	\begin{enumerate}
		\item $\xi_n \cd \xi$
		\item $F_{\xi_n} \cw F_{\xi}$
		\item $\forall x \text{ -- точка непрерывности $F_{\xi}$} : \lim_{n \rightarrow \infty} F_{\xi_n}(x) = F_{\xi}(x)$
	\end{enumerate}
\end{theorem}

\begin{theorem}[многомерный случай, более слабая]
	Пусть $\xi, \xi_1, \xi_2, \dots$ -- случайные векторы из $\mathbb{R}^m$. Пусть $F_{\xi}$ непрерывна. Тогда $\xi_n \cd \xi \Leftrightarrow \forall x \in \mathbb{R}^m : \lim_{n \rightarrow \infty}F_{\xi_n}(x) = F_{\xi}(x)$
\end{theorem}

\subsection{Предельные теоремы}

\begin{theorem}[Закон большых чисел]
	Пусть $\xi_1, \xi_2, \dots$ -- попарно-некоррелированные одинаково распределенные случайные величины, $D\xi_i$ конечна, $S_n = \xi_1 + \dots + \xi_n$. Тогда:
	$$\frac{S_n - ES_n}{n} \cp 0$$, причем $ES_n = n \cdot a$, $a = E\xi_i$.
\end{theorem}

\begin{theorem}[Усиленный закон больших чисел]
	Пусть $\xi_1, \xi_2, \dots$ -- независимые одинаково распределнные случайные величины (или векторы), $E\xi_i = a$ -- конечно, $S_n = \xi_1 + \dots + \xi_n$. Тогда:
	$$\frac{S_n}{n} \cae a$$
\end{theorem}

\begin{theorem}[Центральная предельная теорема]
	Пусть $\xi_1, \xi_2, \dots$ -- независимые одинаково распределнные случайные величины, $E\xi_i = a$ -- конечно, $0 < D\xi_i = \sigma^2$ -- тоже конечно, $S_n = \xi_1 + \dots + \xi_n$. Тогда:
	$$\sqrt{n}\left(\frac{S_n}{n} - a\right) \cd \mathcal{N}(0, \sigma^2)$$
\end{theorem}

\begin{theorem}[Многомерная центральная предельная теорема]
	Пусть $\xi_1, \xi_2, \dots$ -- независимые одинаково распределнные случайные векторы, $E\xi_i = a$ -- конечно, $0 < D\xi_i = \Sigma$ -- матрица ковариаций, тоже конечна, $S_n = \xi_1 + \dots + \xi_n$. Тогда:
	$$\sqrt{n}\left(\frac{S_n}{n} - a\right) \cd \mathcal{N}(0, \Sigma)$$
\end{theorem}

\begin{statement}
	Пусть $\xi, \xi_1, \xi_2, \dots$ -- случайные векторы. Тогда:
	\begin{enumerate}
		\item $\xi_n \cae \xi \Rightarrow \xi_n \cp \xi$
		\item $\xi_n \cp \xi \Rightarrow \xi_n \cd \xi$
	\end{enumerate}
\end{statement}

\begin{proof}
	~\begin{enumerate}
		\item Следует из того, что векторная сходимость эквивалентна покоординатной, а для координат верны одномерные аналоги.
		\item Доказывается аналогично одномерному случаю.
	\end{enumerate}
\end{proof}

\begin{lemma}[о сходящейся подпоследовательности]
	Пусть $\xi_n \cp \xi$. Тогда $\exists \text{ подпоследовательность } \xi_{n_k}: \xi_{n_k} \cae \xi$ 
\end{lemma}

\begin{theorem}[о наследовании сходимостей]
	Пусть $\xi, \xi_1, \xi_2, \dots$ -- случайные векторы из $\mathbb{R}^m$. Пусть также $h: \mathbb{R}^m \mapsto \mathbb{R}^s$ -- функция, непрерывная почти всюду относительно распределения $\xi$ (то есть $\exists B \in \mathcal{B}(\mathbb{R}^m), P_{\xi}(B) = 1: h \text{ непрерывна на $B$}$. Тогда:
	\begin{enumerate}
		\item $\xi_n \cae \xi \Rightarrow h(\xi_n) \cae h(\xi)$
		\item $\xi_n \cp \xi \Rightarrow h(\xi_n) \cp h(\xi)$
		\item Если дополнительно $h$ непрерывна всюду, а не почти всюду, то: $\xi_n \cd \xi \Rightarrow h(\xi_n) \cd h(\xi)$
	\end{enumerate}
\end{theorem}

\begin{proof}
	~\begin{enumerate}
		\item $1 = P(\{\omega : \xi_n(\omega) \rightarrow \xi(\omega)\}) \leq P(\{\omega : h(\xi_n)(\omega) \rightarrow h(\xi)(\omega)\})$, так как
			$\{\omega : \xi_n(\omega) \rightarrow \xi(\omega)\} \subset \{\omega : h(\xi_n)(\omega) \rightarrow h(\xi)(\omega)\}$
		\item Пусть не выполнено, что $h(\xi_n) \cp h(\xi)$. Тогда $\exists \varepsilon, \delta$ а так же подпоследовательность $\xi_{n_k}$: 
			$$\forall k : P(\|h(\xi_{n_k}) - h(\xi)\| \geq \varepsilon) \geq \delta$$. Так как $\xi_{n_k} \cp \xi$, то существует подпоследовательность 
			$\xi_{n_{k_l}}: \xi_{n_{k_l}} \cae \xi$. Тогда согласно первому пункту $h(\xi_{n_{k_l}}) \cae h(\xi)$. 
			Но такого быть не может, т.к. $\forall k : P(\|h(\xi_{n_k}) - h(\xi)\| \geq \varepsilon) \geq \delta$
		\item Пусть $f: \mathbb{R}^m \mapsto \mathbb{R}$ -- непрерывная ограниченная функция. Тогда $g = f \circ h$ -- тоже непрерывная ограниченная функция.
			Так как $\xi_n \cd \xi$, то $Eg(\xi_n) \rightarrow Eg(\xi)$. А значит, $Ef(h(\xi_n)) \rightarrow Ef(h(\xi))$
	\end{enumerate}
\end{proof}
\begin{lemma}[Слуцкого]
Пусть $\xi_n \cd \xi$ -- случайные величины, а $\eta_n \cd C=const$. Тогда $\xi_n + \eta_n \cd \xi + C$ и $\xi_n \cdot \eta_n \cd \xi \cdot C$.
\end{lemma}
\begin{proof}
	Пусть t -- точка непрерывности функции распределения $F_{\xi+C}$ случайной величины $\xi + C$. Докажем только для суммы, для произведения аналогично.
	Пусть $\varepsilon > 0$ такое,что $t \pm \varepsilon$ -- тоже точки непрерывности $F_{\xi + C}$. Мы хотим показать, что $F_{\xi_n+\eta_n}(t) \rightarrow F_{\xi+C}(t)$.
	Будем для этого пользоваться тем, что $\eta_n \cd C \Leftrightarrow \eta_n \cp C$. 
	\begin{multline*}
		P(\xi_n + \eta_n \leq t) = P(\xi_n + \eta_n \leq t, \eta_n \geq C - \varepsilon) + P(\xi_n + \eta_n \leq t, \eta_n < C - \varepsilon) \leq \\
		P(\xi_n \leq t - C + \varepsilon) + P(\|\eta_n - C\| > \varepsilon) = F_{\xi_n + c}(t + \varepsilon) + P(\|\eta_n - C\| > \varepsilon)
	\end{multline*}
	Но $\xi_n \cd \xi \Rightarrow \xi_n + C \cd \xi + C$. Кроме того, $t + \varepsilon$ -- точка непрерывности $F_{\xi + C}$
	\begin{multline*}
		\limsup_{n \rightarrow \infty} F_{\xi_n + \eta_n}(t) \leq \lim_{n \rightarrow \infty} F_{\xi_n + c}(t + \varepsilon) + \lim_{n \rightarrow \infty} P(\|\eta_n - C\| > \varepsilon) = F_{\xi + C}(t + \varepsilon)
	\end{multline*}
	Аналогично,
	$$1 - f_{\xi_n + \eta_n}(t) \leq P(\|\eta_n - C\| > \varepsilon) + 1 - F_{\xi_n + C}(t - \varepsilon)$$
	Откуда
	$$\liminf_{n \rightarrow \infty} F_{\xi_n + \eta_n}(t) \geq F_{\xi + C}(t - \varepsilon)$$
	В силу произвольности $\varepsilon > 0$ и того факта, что $t$ -- точка непрерывности функции распределения $F_{\xi + c}$ получаем, что
	$$\lim_n F_{\xi_n + \eta_n} = F_{\xi + C}(t)$$

\end{proof}

\begin{statement}[применение леммы Слуцкого]
	Пусть $\xi_n \cd \xi$ -- случайные величины. $h(x)$ -- дифференцируема в точке $a \in \mathbb{R}$. $\{b_n > 0\}$ -- числовая последовательность, $b_n \rightarrow 0$. Тогда:
	$$\frac{h(a + \xi_n b_n) - h(a)}{b_n} \cd h'(a) \xi$$
\end{statement}

\begin{proof}
	По лемме Слуцкого $\xi_n b_n \cd \xi_n \cdot 0 = 0$.
	Рассмотрим 
	$$H(x) = \begin{cases}
		\frac{h(x+a) - h(a)}{x}, x \not= 0 \\
		h'(a), x = 0
	\end{cases}$$
	$H(x)$ непрерывна в 0 по определению, а также непрерывна на $\mathbb{R}\setminus 0$ как композиция непрерывных функций. По теореме о наследовании сходимостей
	$H(\xi_n b_n) \cd H(0) = h'(a)$. По лемем слуцкого $\xi_n H(\xi_n b_n) \cd h'(a) \xi$
\end{proof}

\begin{theorem}[многомерный вариант]
	Пусть $\xi_n \cd \xi$ -- случайные векторы. $h(x)$ -- дифференцируема в точке $a \in \mathbb{R}$ (то есть существует матрица частных производных, или матрица Якоби $J(h)$). $\{b_n > 0\}$ -- числовая последовательность, $b_n \rightarrow 0$. Тогда:
	$$\frac{h(a + \xi_n b_n) - h(a)}{b_n} \cd J(h)(a) \xi$$
\end{theorem}
\end{document}

