\documentclass[document.tex]{subfiles}

\begin{document}

\section{Сравнение оценок}
\begin{definition}
    Пусть $X$ -- наблюдение с неизвестным распределением $P \in \{P_{\theta}: \theta \in \Theta\}$. $\rho(x, y)$ --
    функция потерь. Тогда функцией риска оценкаи $\hat \theta(X)$ неизвестного параметра $\theta$ называется:
    $R(\hat \theta(X), \theta) = E_{\theta} \rho(\hat \theta(X), \theta)$
\end{definition}

\subsection{Равномерный подход}

\begin{definition}
    Оценка $\hat \theta(X)$ называется лучшей оценки $\theta^*(X)$ в равномерном подходе, если $\forall \theta \in
    \Theta: R(\hat \theta(X), \theta) \leq R(\theta^*(X), \theta)$ и для некоторого $\theta \in \Theta$ неравенство
    строгое.
\end{definition}

\begin{definition}
    Если оценка $\hat \theta(X)$ лучше любой другой оценки в каком-либо классе оценок, то она называется наилучшей в
    этом классе
\end{definition}

\begin{remark}
    Равномерный подход с квадративной функцией потерь называется среднеквадратическим подходом. Не для любого класса
    можно отыскать наилучшую оценку.
\end{remark}

\begin{definition}
    $K$ -- несмещенные оценки $\tau(\theta)$. В таком классе $K$ со среднеквадратичной функцией потерь:
    \[
        R(\hat \theta(X), \theta) = E_{\theta} (\hat \theta(X) - \theta)^2 = E_{\theta} (\hat \theta(X) - \tau(Theta))^2
        + (\tau(\theta) - \theta)^2 = D_{\theta} \hat \theta(X)
    \]
\end{definition}

\begin{definition}
    Оценка называется допустимой, если для неё не существует лучшей оценки в равномерном подходе.
\end{definition}

\subsection{Байесовский подход}

Пусть $R(\hat \theta, \theta)$ -- функция риска для оценки $\hat \theta$, и задано $Q$ -- нек. распределение
вероятностей на $\Theta$. Тогда можно определить $R(\hat \theta) = \int_{\Theta} R(\hat \theta, t) Q(dt)$. Если $Q$
имеет плотность $q(t)$, то $R(\hat \theta) = \int_{\Theta} R(\hat \theta, t) q(t) dt$

\begin{definition}
    Если $R(\hat \theta) = \min_{\theta^* \in K} R(\theta^*)$, то $\hat \theta$ называется наилучшей в байесовском
    подходе в классе $K$.
\end{definition}

Байесовские оценки являются допустимыми.

\subsection{Минимаксный подход}

Если $\hat R(\hat \theta) = \min_{\theta^* \in K} \sup_{\theta \in \Theta} R(\theta^*, \theta)$, то $\hat \theta$
называется наилучшей в минимаксном подходе в классе $K$.

\subsection{Асимптотический подход}
$X = (X_1, \cdots, X_n)$ -- выборка растущего размера.

Пусть $\hat \theta_1$, $\hat \theta_2$ -- две асимптотические оценки $\theta \in \Theta \subset \mathbb{R}$. Пусть
$\sigma_1^2(\theta)$, $\sigma_2^2(\theta)$ -- их асимптотические оценки. Мы будем говорить, что $\hat \theta_1$ лучше
$\hat \theta_2$, если $\forall \theta \in \Theta: \sigma_1^2(\theta) \leq \sigma_2^2(\theta)$ и для некоторого $\theta
\in \Theta$ неравенство строое.

Оценка называется наилучшей в асимптотическом подходе в каком-то классе, если она лучше любой другой оценки в каком-то
классе. 

\begin{example}
    $X = (X_1, \cdots, X_n)$ -- выборка из номального распределения с параметрами $(\theta, 1)$. Нужно сравинить в
    асимптотическом подходе оценки $\overline X$ и $\hat \mu$.

    $\sqrt{n}(\overline X - \theta) \sim N(0, 1)$ по ЦПТ.

    По теореме об асимптотической нормальности выборочного квантиля:

    $\sqrt{n}(\hat \mu - \theta) \sim N(0, \frac{1}{1/4 \cdot p(\theta)^2} = N(0, \frac{\pi}{2})$.

    Получили, что $\overline X$ лучше $\hat \mu$
\end{example}

\subsection{Понятие плотности дискретного распределения}

Функция $p(x) \geq 0$ называется плотностью вероятностной меры $P$ на $(\mathbb{R}, \mathcal{B}(\mathbb{R}))$, если
$\forall B \in \mathcal{B}(\mathbb{R}): P(B) = \in_{B}p(x)dx$. В этом случае $P$ называется абсолютно непрерывной
вероятностной мерой, $p(x)$ -- плотность по мере Лебега.

\begin{definition}
    Функция $\mu: \mathcal{B}(\mathbb{R}) \mapsto \mathbb{Z} \cup \{\infty\}$ определенная по правилу: 
    \[
        \mu(B) = \sum_{k \in \mathbb{Z}} I\{k \in B\} 
    \] называется считающей мерой на $\mathbb{Z}$.
\end{definition}

\begin{definition}
    Интегралом от функции $f(x)$ по считающей мере $\mu$ называется $\int_{\mathbb{R}} f(x) \mu(dx) := \sum_{k \in
        \mathbb{Z}}f(k)$
\end{definition}
Для такого интеграла выполнены все основные свойства: линейность, сохранение отношения порядка, теоремы о сходимости и
так далее. Аналогично можно определить считающую меру в $\mathbb{Z}^n$ и интеграл по ней.

\begin{definition}
    Пусть $\xi$ -- случайная величина со значениями в $\mathbb{Z}$. Тогда её плотность по считающей мере $\mu$
    называется $p(x) = P(\xi = x)$ 
\end{definition}

\begin{corollary}
    Для любой функции $g(x)$ выполнено $Eg(\xi) = \int_{\mathbb{R}} g(x) p(x) \mu(dx)$
\end{corollary}

\begin{definition}
    Пусть $X$ -- некоторое наблюдение с неизвестным распределением $P \in \{P_{\theta} : \theta \in \Theta\}$. Если
    $\forall \theta \in \Theta: P_{\theta}$ имеет плотность $p_{\theta}$ по одной и той же мере (либо мере Лебега, либо
    по считающей мере), то в этом случае $\{P_{\theta}: \theta \in \Theta\}$ называется доминируемым семейством.
\end{definition}

\begin{remark}
    Для меры всегда будем использовать единое обозначение $\mu$
\end{remark}

\subsection{Неравенство Рао-Крамера и эффективные оценки}
Пусть $X$ -- наблюдение с неизвестным распределением $P \in \{P_{\theta} : \theta \in \Theta\}$ -- доминируемое
семейство с плотностью $p_{\theta}$. Предположим, что выполнены следующие условия регулярности:

~\begin{enumerate}
    \item $A = \{x : p_{\theta}(x) > 0\}$ не зависит от параметра $\theta$

    \item $\Theta$ -- открытый интервал на $\mathbb{R}$ (может быть бесконечный)

    \item $\forall S(x): E_{\theta} (S(X))^2 < \infty$ выполнено $\frac{\partial}{\partial \theta} E_{\theta}S(X) =
        E_{\theta}(S(X) \frac{\partial}{\partial \theta} \ln p_{\theta}(X))^2$

    \item Интеграл $I_X(\theta) = E_{\theta} (\frac{\partial}{\partial \theta}\ln p_{\theta}(x))^2$ положителен и
        конечен $\forall \theta \in \Theta$
\end{enumerate}

\begin{definition}
    Случайная величина $U_{\theta}(X) = \frac{\partial}{\partial \theta}\ln p_{\theta}(X)$ называется вкладом в
    наблюдение $X$. $I_X(\theta) = E_{\theta}(U_{\theta}(X))^2$ называется количеством информации (по Фишеру),
    содержащейся в наюлюдении $X$
\end{definition}

\begin{theorem}[Неравенство Рао-Крамера]
    В условиях регулярности, если $\hat \theta(X)$ -- несмещенная оценка $\tau(\theta)$, причем $E_{\theta}(\hat
    \theta(X))^2$ конечен $\forall \theta$. Тогда выполнено следующее неравенство:
    $D_{\theta}\hat \theta(X) \geq \frac{(\tau'(\theta))^2}{I_{X}(\theta)}$
\end{theorem}

\begin{proof}
    Положим $S(x) \equiv 1$. В условии 3 получим $0 = E_{\theta}U_{\theta}(X)$. Возьмем теперь $S(x) = \hat \theta(X)$ в
    условии 3: $\frac{\partial}{\partial \theta} \hat \theta(X) = \tau(\theta) = E_{\theta}\hat \theta(X)
    U_{\theta}(X)$. То есть имеем: $\tau'(\theta) = E_{\theta}\hat \theta(X) U_{\theta}(X)$. Умножим первое равенство на
    $\tau(\theta)$ и вычтем из второго. Получим:

    $\tau'(\theta) - 0 = E_{\theta}(\hat \theta - \tau(\theta))U_{\theta}(X)$ Применяем КБШ. Получаем, что 
    $\tau'(\theta)^2 \leq D_{\theta}\hat \theta \cdot I_X(\theta)$
\end{proof}

\begin{corollary}
    Если $\tau(\theta) = \theta$, то $D_{\hat \theta} \geq \frac{1}{I_X(\theta)}$
\end{corollary}

Если $X = (X_1, \cdots, X_n)$ -- выборка, то $I_X(\theta) = n \cdot i(\theta)$, где $i(\theta)$ -- информация одного
наблюдения. В этом случае $D_{\theta}\hat \theta = \Omega(\frac{1}{n})$

\begin{definition}
    Оценки, в которых в неравенстве Рао-Крамера достигается равенство, называются эффективными.
\end{definition}


\end{document}

