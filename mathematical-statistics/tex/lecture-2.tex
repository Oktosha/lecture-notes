\documentclass[document.tex]{subfiles}

\begin{document}
\section{Вероятностно-статистические модели и выборки}
\subsection{Вероятностно-статистическая модель}

\begin{definition}
	Множество всех возможных значений наблюдения называется выборочным пространством и обозначается $\mathfrak{X}$
\end{definition}

\begin{definition}
	Наблюдение $X$ -- это результат случайного выбора элемента из выборочного пространства. Наша цель -- по наблюдению $X$ сделать выводы о его распределении $P$.
\end{definition}

\begin{definition}
	Если $X = (X_1, \dots, X_n)$ -- набор независимых одинаково-распределенных случайных величин имеющих распределению $P$, то $X$ называется выборкой размера $n$ из неизвестного распределения $P$
\end{definition}

\begin{remark}
	В дальнейшем будем обозначать: $X = (X_1, \dots, X_n)$ -- выборка на вероятностном пространстве $(\Omega, \mathcal{F}, P)$ c неизвестным распределением $P$ на $(\mathfrak{X}, \mathcal{B}_x)$ (например, $(\mathbb{R}, \mathcal{B}(\mathbb{R}))$).
\end{remark}

\begin{definition}
	Для каждого множества $B \in \mathcal{B}_x)$ введем $P_n^*(B) = \frac{\nu_n(B)}{n}$, где $\nu_n(B)$ -- это количество элементов из $X_1, \dots, X_n$, которые попали в $B$. То есть формально:
	$$P_n^*(B) = \frac{1}{n}\sum_{i = 1}^n I\{X_i \in B\}$$.
	Такая величина называется эмпирическим распределением.
\end{definition}

\begin{statement}
	Пусть $P$ -- неизвестное распределение $X_i$. Тогда $\forall B \in \mathcal{B}_x: P_n^*(B) \cae P(B)$.
\end{statement}

\begin{proof}
	Фиксируем $B \in \mathcal{B}_x$. Тогда для фиксированного $B$ индикаторы $I\{X_i \in B\}$ будут являтся случайными величинами, причем независимыми и
	одинаково распределенными, поскольку исходные случайные величины были независимыми и одинаково распределенными.
	Введем 
	$$S_n = \sum_{i = 1}^n I\{X_i \in B\}$$
	По усиленному закону больших чисел: 
	$$\frac{1}{n}S_n \cae P(X_1 \in B) = P(B)$$
\end{proof}

\begin{definition}
	Пусть $X = (X_1, \dots, X_n$) -- выборка. 
	$$F_n(x) = \frac{1}{n} \sum_{i = 1}^n I\{X_i \leq x\}$$
	называется эмпирической функцией распределения (она является функцией распределения для эмпирического распределения $P_n^*$).
\end{definition}

\begin{remark}
	$$\forall x \in \mathbb{R}: F_n(x) \cae F(x)$$
\end{remark}

\begin{theorem}[Гливенко - Кантелли]
	Если $F(x)$ -- функция распределения элементов выборки $X_1, \dots, X_n$ (то есть функция распределения для распределения $P$), то:
	$$\sup_{x \in \mathbb{R}} |F_n(x) - F(x)| \cae 0$$
\end{theorem}

\begin{proof}
	Зафиксируем элементарный исход $\omega \in \Omega$. Тогда случайные величины $X_1, \dots, X_n$ превращаются в числа. Посмотрим на функцию распределения $F_n(x)$. Она является непрерывной справа, так как является конечной суммой индикаторов вида $I\{X_i \leq x\}$, а $F(x)$ непрерывна справа как функция распределения. Модуль их разности тоже непрерывен справа. Тогда $\sup_{x \in \mathbb{R}} |F_n(x) - F(x)| = \sup_{x \in \mathbb{Q}} |F_n(x) - F(x)|$ -- это супремум счетного числа случайных величин. Поэтому $|F_n(x) - F(x)|$ тоже является случайной величиной.
	
	Пусть $N \in \mathbb{N}$ -- достаточно большое. Для каждого $k \in \{1, \dots, N - 1\}$ введем $x_{k, N} = \inf \{x : F(x)  \geq \frac{k}{N}\}$. Полагаем также $x_{0, N} = -\infty, x_{N, N} = +\infty$. Оценим $F_n(x) - F(x)$ для $x \in [x_{k, N}, x_{k+1, N})$:
	\begin{multline*}
		F_n(x) - F(x) \leq F_n(x_{k+1, N} - 0) - F(x_{k, N}) = \\
		= F_n(x_{k+1, N} - 0) - F(x_{k+1, N} - 0) + F(x_{k+1, N} - 0) - F(x_{k, N}) \leq \\
		F_n(x_{k+1, N} - 0) - F(x_{k+1, N} - 0) + \frac{1}{n}
	\end{multline*}
	Совершенно аналогично показывается, что $F_n(x) - F(x) \geq F_n(x_{k, N}) - F(x_{k, N}) - \frac{1}{N}$
	Откуда получаем, что:
	$$\sup_{x \in \mathbb{Q}} |F_n(x) - F(x)| \leq \max_{k, l} \{ |F_n(x_{k, N} - 0) - F(x_{k, N} - 0)|, |F_n(x_{l, N}) - F(x_{l, N})|\} + \frac{1}{N}$$
	Зафиксируем $\varepsilon > 0$ и возьмем $N : \frac{1}{N} < \varepsilon$. По усиленному закону больших чисел $F_n(x) \cae F_(x)$, $F_n(x - 0) \cae F(x - 0)$.
	То есть $\forall x \in \mathbb{Q} : P(\limsup_{n} |F_n(x) - F(x)| > \varepsilon) = 0$. 
	А значит, $P(\sup_{x \in \mathbb{Q}} \limsup_{n} |F_n(x) - F(x)| > \varepsilon) = 0$. 
	Поменяв местами $\sup$ и $\limsup$ и считая $\varepsilon = \frac{1}{m}$, получаем:
	$$\forall m: P(\limsup_{n} \sup_{x \in \mathbb{Q}} |F_n(x) - F(x)| > \frac{1}{m}) = 0$$
	Пользуясь теоремой о непрерывности вероятностной меры, имеем:
	$$P(\lim_{n} \sup_{x \in \mathbb{Q}} |F_n(x) - F(x)| > 0) = 0$$
\end{proof}

\begin{remark}
	Пусть на $\mathfrak{X}$ задана $\sigma$-алгебра $\mathcal{B}_x$. Как правило $(\mathfrak{X}, \mathcal{B}_x) = (\mathbb{R}, \mathcal{B}(\mathbb{R}))$
	Наблюдение $X$ -- это, формально, тождественная случайная величина на $(\mathfrak{X}, \mathcal{B}_x, P)$. Обычно про распределение $P = P_X$ известно,
	что оно пренадлежит некому классу распределений $\mathcal{P}$, например, классу нормальных распределений
\end{remark}

\begin{definition}
	Тройка $(\mathfrak{X}, \mathcal{B}_x, \mathcal{P})$, где $\mathcal{P}$ -- это класс вероятностных мер на $(\mathfrak{X}, \mathcal{B}_x)$, называется
	вероятностно-статистической моделью
\end{definition}

\begin{remark}
	$\forall P \in \mathcal{P}: (\mathfrak{X}, \mathcal{B}_x, P)$ -- вероятностное пространство
\end{remark}

\begin{definition}
	Вероятностно-статистическая модель $(\mathfrak{X}, \mathcal{B}_x, \mathcal{P})$ называется параметрической, еcли класс $\mathcal{P}$ параметризован,
	то есть $\mathcal{P} = \{P_{\theta} : \theta \in \Theta\}$. Также считаем, что $P_{\theta_1} \neq P_{\theta_2}$, если $\theta_1 \neq \theta_2$
\end{definition}

\subsection{Моделирование выборки}
\subsubsection{Конечная выборка}
Мы хотим смоделировать конечную выборку $(X_1, \dots, X_n)$ в терминах веротяностно-статистической модели. Пусть $X_i$ принимает значения из $\mathfrak{X}$ и имеет
неизвестное распределение $P \in \mathcal{P}$. В этом случае удобно рассмотреть следующую статистическую модель: $(\mathfrak{X}^n, \mathcal{B}_x^n, \mathcal{P}^n)$, где $\mathfrak{X}^n = \mathfrak{X} \times \ldots \times \mathfrak{X}$, $\mathcal{B}_x^n = \sigma(B_1 \times \ldots \times B_n : \forall i : B_i \in \mathcal{B}_x)$,
$\mathcal{P}^n = \{P^n : P \in \mathcal{P}\}$, $P^n(B_1 \times \ldots \times B_n) = P(B_1) \cdot \ldots \cdot P(B_n)$. То есть в качестве выборочного пространства мы берем декартову степень, в качестве сигма алгебры, как и в случае с $\mathcal{B}(\mathbb{R}^n)$, минимальную сигма-алгебру, порожденную декартовыми произведениями всех измеримых множеств, а в качестве класса мер -- класс степеней всех мер, где степень меры означает естественное продолжение одномерной меры на многомерное пространство. Какими же выбрать $(X_1, \dots, X_n)$? Это просто: $X_i : \mathfrak{X}^n \mapsto \mathfrak{X}$ такое что $X_i((x_1, \dots, x_n)) = x_i$.
\subsubsection{Счетная выборка}
TO BE CONTINUED\ldots

\begin{remark}
	Мы будем опускать индексы у $\mathfrak{X}$, $\mathcal{B}_x$, $\mathcal{P}$ в целях удобства
\end{remark}

\begin{remark}
	В параметрической модели вопрос о неизвестном распределении $P_{\theta}$ сводится в вопросу о значении $\theta \in \Theta$
\end{remark}

\subsection{Статистики и оценки}
Пусть $X$ -- наблюдение со значениями из $\mathfrak{X}$ и неизвестным распределением $P_{\theta}$, где $\theta \in \Theta$

\begin{definition}
	Статистикой $S(X)$ называется измеримая функция от результатов наблюдения, то есть:
	$S : \mathfrak{X} \mapsto E$, где $(E, \mathcal{E})$ -- измеримое пространство, а $S$ является $\mathcal{E} | \mathcal{B}_x$ измеримой.

	Если $S : \mathfrak{X} \mapsto \Theta$, то $S$ называется оценкой параметра из $\Theta$.
\end{definition}

\begin{example}
	~\begin{enumerate}
		\item Если $g: \mathbb{R} \mapsto \mathbb{R}$ -- борелевская функция, то $\overline{g(x)} = \frac{1}{n} \sum_{i = 1}^n g(X_i)$ называется выборочной
			харакетристикой функции $g(x)$ (среднее значение по элементам выборки)
	\end{enumerate}
\end{example}

\end{document}

