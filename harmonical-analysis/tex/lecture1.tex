\documentclass[document.tex]{subfiles}

\begin{document}
\section{Абсолютно интегрируемые функции}
\begin{definition}
Функция $f: [a, b] \mapsto \mathbb{R}$ называется ступенчатой, если $\exists \{c_i\}_{i=0}^n,
a = c_0 < c_1 < \ldots < c_n = b: \forall i \in \{1, \ldots, n\}: f\text{ -- постоянная на }(c_{i-1}, c_i)$.
\end{definition}

\begin{theorem}
Пусть $f : [a, b] \mapsto \mathbb{R}$ интегрируема по Риману. Тогда $\forall \varepsilon > 0: \exists h(x)\text{ -- ступенчатая} : \int_a^b|f(x)-h(x)|dx < \varepsilon$
\end{theorem}
\begin{proof}
Пусть $\tau = \{c_i\}_{i=0}^n$ -- некоторое разбиение отрезка $[a, b]$, т.е. $a = c_0 < c_1 < \ldots < c_n = b$. Пусть $\xi_i \in (c_{i - 1}, c_i)$. Запишем сумму Римана для данного разбиения:
$$\sum_{i=1}^n f(\xi_i)(c_i - c_{i-1})$$. Рассмотрим ступенчатую функцию
 $$h(x) = \begin{cases}
 	f(\xi_i),\text{ if } x \in (c_{i-1}, c_i),\\
 	0, \text{ elsewhere}.
\end{cases}$$
Рассмотрим интеграл:
$$\int_a^b|f(x)-h(x)|dx = \sum_{i=1}^n\int_{c_{i-1}}^{c_i}|f(x) - f(\xi_i)|dx \leq$$
$$\leq \sum_{i=1}^n\omega_i(f)(c_i - c_{i-1})$$,
где $\omega_i(f)$ -- это колебание функции $f$ на отрезке $[c_{i-1}, c_i]$. По критерию интегрируемости последняя сумма меньше $\varepsilon$ для любого такого $\varepsilon$ для всех достаточно мелких разбиений.
\end{proof}

\begin{definition}
Пусть функция $f: (a, b) \mapsto \mathbb{R}$, где $-\infty \leq a < b \leq +\infty$. Такая функция называется абсолютно интегрируемой, если:
\begin{enumerate}
\item $\exists \{c_i\}_{i=0}^n, a = c_0 < c_1 < \ldots < c_n = b:$ f интегрируема на любом отрезке $[\alpha, \beta] \subset (a, b) \setminus \{c_i\}$
\item $|f|$ интегрируема в несобственном смысле на $(a, b)$
\end{enumerate}
\end{definition}

\begin{definition}
Функция $f: \mathbb{R} \mapsto \mathbb{R}$ называется финитной, если она равна нулю вне некоторого отрезка, т.е. $\exists [a, b] : \forall x \not \in [a, b] : f(x) = 0$.
\end{definition}

\begin{definition}
Функция $f : \mathbb{R} \mapsto \mathbb{R}$ называется финитно-ступенчатой, если она финитная и ступенчатая.
\end{definition}

\begin{theorem}
Пусть функция $f : [a, b] \mapsto \mathbb{R}$ -- абсолютно интегрируема. Тогда $\forall \varepsilon > 0: \exists h(x)\text{ -- финитно-ступенчатая} : \int_a^b|f(x)-h(x)|dx < \varepsilon$
\end{theorem}
\begin{proof}
Пусть $\varepsilon > 0$.
Можно считать, что $a = -\infty$, $b = +\infty$, иначе доопределим $f$ на $\mathbb{R} \setminus [a, b]$ нулем.
Поскольку $f$ -- абсолютно интегрируема, то:

\begin{enumerate}
\item $\exists \{c_i\}_{i=0}^n, a = c_0 < c_1 < \ldots < c_n = b:$ f интегрируема на любом отрезке $[\alpha, \beta] \subset (a, b) \setminus \{c_i\}$
\item $|f|$ интегрируема в несобственном смысле на $(a, b)$
\end{enumerate}
$$\exists \{a_i\}_{i=1}^n, \{b_i\}_{i=1}^n, -\infty < a_1 < b_1 < c_1 < a_2 < b_2 < c_2 < \ldots < a_n < b_n < +\infty:$$
$$\int_{-\infty}^{a_1}|f(x)|dx + \int_{b_1}^{a_2}|f(x)|dx + \ldots + \int_{b_n}^{+\infty}|f(x)|dx < \varepsilon$$, т.к. $|f|$ интегрируем на $\mathbb{R}$ в несобственном смысле. Пусть
$$f_{\varepsilon}(x) = \begin{cases}
f(x), x \in \cup [a_i, b_i], \\
0, elsewhere.
\end{cases}$$
Тогда $$\int_{-\infty}^{+\infty}|f(x) - f_{\varepsilon}(x)|dx < \varepsilon$$.
По предыдущей теореме существует $h_{\varepsilon}(x):[a_1, b_n] \mapsto \mathbb{R}$ -- ступенчатая функция: $\int_{a_1}^{b_n}|h_{\varepsilon}(x) - f_{\varepsilon}(x)|dx < \varepsilon$. Доопределим $h_{\varepsilon}(x)$ нулем вне отрезка $[a_1, b_n]$. Тогда $h$ -- финитно-ступенчатая. Кроме того:
$$\int_{-\infty}^{+\infty}|h_{\varepsilon}(x) - f_{\varepsilon}(x)|dx < \varepsilon$$
Тогда:
$$\int_{-\infty}^{+\infty}|h_{\varepsilon}(x) - f(x)|dx < \int_{-\infty}^{+\infty}(|h_{\varepsilon}(x) - f_{\varepsilon}(x)| + |f_{\varepsilon}(x) - f(x)|)dx < 2\varepsilon$$
\end{proof}
\begin{theorem}
Пусть $f: (a, b) \mapsto \mathbb{R}$ абсолютно интегрируема. Доопределим её на $\mathbb{R}$ нулем. Тогда $$\int_{-\infty}^{+\infty} |f(x + \alpha) - f(x)|dx \rightarrow 0, \text{ при } \alpha \rightarrow 0$$
\end{theorem}
\begin{proof}
Этот интеграл существует, т.к. его можно оценить как $\int_{-\infty}^{+\infty}|f(x+\alpha)|dx + \int_{-\infty}^{+\infty}|f(x)|dx$ (см. предыдущее доказательство).

Докажем эту теорему для произвольной финитно-ступенчатой функции $h: \mathbb{R} \mapsto \mathbb{R}$. Поскольку $\alpha \rightarrow 0$, мы можем рассматривать лишь такие $\alpha$, что $$|\alpha| < \frac{\min_i (c_i - c_{i-1})}{2}$$. Пусть $M = \sup |h|$. Тогда 
$$\int_{-\infty}^{+\infty} |h(x + \alpha) - h(x)|dx < 2M(n+1)|\alpha|$$. Пусть $h(x)$ -- финитно-ступенчатая: $\int_{-\infty}^{+\infty}|f(x) - h(x)|dx < \varepsilon$. Тогда:
$\int |f(x+\alpha) - f(x)|dx \leq \int|f(x+\alpha)-h(x+\alpha)|dx + \int|h(x+\alpha)-h(x)|dx + \int|h(x)-f(x)|dx \leq 3\varepsilon$.
\end{proof}

\begin{lemma}[Римана об осцилляции]
Пусть $f : (a, b) \mapsto \mathbb{R}$ -- абсолютно интегрируемая функция. Тогда 
$$\lim_{\lambda \rightarrow \infty} \int f(x)\sin (\lambda x)dx = \lim_{\lambda \rightarrow \infty} \int f(x)\cos (\lambda x)dx = 0$$
\end{lemma}
\begin{proof}
Доопределим $f$ нулем на $\mathbb{R}$. Рассмотрим:
$$I(\lambda) = \int_{-\infty}^{+\infty}f(x)\sin(\lambda x)dx$$
Этот интеграл существует, так как $f$ абсолютно интегрируема. Сделаем замену $x = t + \frac{\pi}{\lambda}$.
Тогда:
$$I(\lambda) = -\int_{-\infty}^{+\infty}f(t + \frac{\pi}{\lambda})\sin(\lambda t)dt$$.
Сложив и поделив пополам, получаем, что:
$$I(\lambda) = \frac{1}{2} \int_{-\infty}^{+\infty}(f(x) - f(x+\frac{\pi}{\lambda})\sin(\lambda x)dx$$
Применяя теорему, которую мы недавно доказали, получаем, что $$I(\lambda) \rightarrow 0$$.
\end{proof}

\end{document}