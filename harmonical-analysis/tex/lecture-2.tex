\documentclass[document.tex]{subfiles}

\begin{document}
\section{Тригонометрический ряд Фурье. Ядро Дирихле. Принцип локализации}
\subsection{Тригонометрический ряд Фурье}
\begin{definition}
	Функциональный ряд
	\begin{equation}
		\label{eq:fourier}
        \frac{a_0}{2} + \sum_{k=1}^{\infty} a_k \cos(kx) + b_k \sin(kx)
	\end{equation} называется тригонометрическим рядом. $a_k, b_k$ -- коэффициенты ряда.
	Система 
    \[
        \{\frac{1}{2}, \sin x, \cos x, \sin 2x, \cos 2x, \ldots\}
    \]
    называется тригонометрической системой
\end{definition}

\begin{statement}
	Тригонометрическая система является ортогональной со скалярным произведением $[f, g] = \int_{-\pi}^{\pi}f(x)g(x)dx$
\end{statement}
\begin{proof}
    \begin{multline*}
        \\
        [1, 1] = \int_{-\pi}^{\pi}dx \neq 0 \\
        [\sin kx, \sin kx] = \int_{-\pi}^{\pi}\sin^2 kx dx \neq 0 \\
        [\cos kx, \cos kx] = \int_{-\pi}^{\pi}\cos^2 kx dx \neq 0 \\
        [1, \sin kx] = \int_{-\pi}^{\pi}\sin kx dx = 0 \\
        [1, \cos kx] = \int_{-\pi}^{\pi}\cos kx dx = 0 \\
        [\sin nx, \cos mx] = \int_{-\pi}^{\pi}\sin nx \cos mx dx = 0 \\
        [\sin nx, \sin mx] = \frac{1}{2} \int_{-\pi}^{\pi}\cos (n - m)x - \cos (n + m)x dx = 0 \\
        [\sin nx, \sin mx] = \frac{1}{2} \int_{-\pi}^{\pi}\cos (n - m)x + \cos (n + m)x dx = 0 \\
    \end{multline*}
    Заметим, что второе и третье равенство выполнены, так как интеграл берется от положительной непрерывной функции.
    Четвертое и пятое получается использованием Формулы Ньютона-Лейбницы. Шестое выполенно, так как подинтегральная
    функция нечетна. А седьмое и восьмое (при $n \neq m$) -- это результат применения тригонометрических Формул и использовании формулы
    Ньютона-Лейбница.
\end{proof}

\begin{definition}
	Пусть функция $f$ является $2 \pi$ периодической и абсолютно интегрируемой на $[-\pi, \pi]$. Тогда ряд
    \ref{eq:fourier}, где
    $$b_k = \frac{1}{\pi}\int_{-\pi}^{\pi}f(x) \sin kx dx$$
	$$a_k = \frac{1}{\pi}\int_{-\pi}^{\pi}f(x) \cos kx dx$$
    называется тригонометрическим рядом Фурье функции $f$. Обозначение:
    Обозначаем $f \sim \frac{a_0}{2} + \sum_{k=1}^{\infty} a_k \cos(kx) + b_k \sin(kx)$.
\end{definition}

\begin{statement}
	Пусть $$T(x) = \frac{\alpha_0}{2} + \sum_{i=0}^{\infty} \alpha_k \cos kx + \beta_k \sin kx$$ сходится равномерно на $\mathbb{R}$. Тогда ряд Фурье для $T(x)$ -- это сам $T(x)$.
\end{statement}
\begin{proof}
    Зафиксируем $k \in \mathbb{N}$. Мы хотим показать, что 
    \[
        a_k = \int_{-\pi}^{\pi}T(x) \cos kxdx = \alpha_k
    \]
	Обозначим за $T_n(x)$ сумму первых $n$ членов ряда.
	Фиксируем $\varepsilon > 0$. Так как ряд $T(x)$ сходится равномерно, то $\exists n > k: |T(x) - T_n(x)| < \varepsilon$.

	$$\pi a_k = \int_{-\pi}^{\pi} T(x)\cos kx dx = \int_{-\pi}^{\pi} (T_n(x) + (T(x) - T_n(x))) \cos kx dx = $$
	$$= \int_{-\pi}^{\pi} T_n(x) \cos kx dx + \int_{-\pi}^{\pi} (T(x) - T_n(x)) \cos kx dx$$
    Заметим, что первое слагаемое в этой сумме равно $\alpha_k$.
	Тогда $$\pi |a_k - \alpha_k| = \left| \int_{-\pi}^{\pi}(T(x) - T_n(x)) \cos kx dx \right| \leq 2 \pi \varepsilon$$
\end{proof}
\begin{corollary}[из леммы Римана об осцилляции]
	Если $f$ -- абсоллютно интегрируемая и $2 \pi$ периодическая, то $a_k \rightarrow 0, b_k \rightarrow 0$ при $n \rightarrow \infty$.
\end{corollary}

\subsection{Ядро Дирихле}

\begin{definition}
	$$S_n(f)(x) = \frac{a_0}{2} + \sum_{k = 1}^n a_k \cos kx + b_k \sin kx$$
\end{definition}

\begin{definition}
	$D_n(x) = \frac{1}{2} + \sum_{k=1}^n \cos kx$ -- ядро Дирихле.
\end{definition}

\begin{remark}
	Умножим и поделим ядро Дирихле на $\sin x/2$. Получаем:
	$$\sin \frac{x}{2} D_n(x) = \sum_{k = 1}^n \frac{\sin(k + \frac{1}{2})x + \sin (k - \frac{1}{2})x}{2}
	= \frac{\sin(n+\frac{1}{2})x}{2}$$
	Можно считать, что $$D_n(x) = \frac{\sin(n+\frac{1}{2})x}{2 \sin \frac{x}{2}}$$, если $x \neq 0$, а в нуле $D_n(x)$
    -- это предел этого выражения, то есть $n + \frac{1}{2}$
\end{remark}

\begin{remark}
	$$\frac{1}{\pi} \int_{-\pi}^{\pi} D_n(x) dx = 1$$
\end{remark}

\begin{remark}
	Подставим в формулу для $S_n(f)(x)$ коэффициенты $a_k$ и $b_k$. После использования формулы косинуса разности, а так
    же вспоминая, что ядро Дирихле -- чётная функция, получаем:
    \begin{multline*} 
        S_n(f)(x) = \frac{1}{\pi} \int_{-\pi}^{\pi} D_n(t - x) f(t) dt = \frac{1}{\pi} \int_{-\pi - x}^{\pi - x}D_n(u) f(x + u)
        du = \\
        \frac{1}{\pi}\int_{-\pi}^{\pi}D_n(u) f(x + u) du = 
        \frac{1}{\pi} \int_{-\pi}^{0}D_n(u) f(x + u) du + \frac{1}{\pi} \int_{0}^{\pi}D_n(u) f(x + u) du = \\
        \frac{1}{\pi} \int_{0}^{\pi}D_n(-u) f(x - u) du + \frac{1}{\pi} \int_{0}^{\pi}D_n(u) f(x + u) du = \\
        \frac{1}{\pi} \int_{0}^{\pi}D_n(u) (f(x - u) + f(x + u)) du = 
        \frac{1}{\pi} \left(\int_0^{\delta} + \int_{\delta}^{\pi} \right) D_n(t) (f(x+t) + f(x-t))dt = \\
        \frac{1}{\pi} \int_{0}^{\delta}D_n(t) (f(x+t) + f(x - t)) dt + \int_{\delta}^{\pi}\frac{\sin (n + 1/2)t}{2\sin
        t/2} (f(x+t) + f(x-t))dt
    \end{multline*}
    В последнем интеграле функция $\frac{f(x + t) + f(x - t)}{\sin t/2}$ абсолютно интегрируема, если исходная функция
    абсолютно интегрируема, а значит, по лемме
    Римана об осцилляции, последний интеграл стемится к нулю.
\end{remark}

\subsection{Принцип локализации}

\begin{theorem}
	Пусть $f$ -- $2\pi$ периодическая, абсолютно интегрируемая на $[-\pi, \pi]$ функция. Тогда частичные суммы $S_n(f)(x)$ сходятся в точке
	$x_0$ тогда и только тогда, когда $\exists \delta > 0$, что сходится интеграл $$\int_0^{\delta} D_n(t) (f(x+t) + f(x-t))dx$$. Причем если они сходятся, то к одному
	и тому же числу.
\end{theorem}

\begin{corollary}[принцип локализации]
	Сходимость ряда Фурье в точке и величина предела зависят только от значений функции в сколь угодно малой окрестности этой точки.
\end{corollary}

\end{document}

