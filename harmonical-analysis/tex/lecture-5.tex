\documentclass[document.tex]{subfiles}

\begin{document}
\section{Метрические и нормированные пространства. Банаховы пространства. Евклидовы пространства. Неравенство
треугольника и Коши-Буняковского для Евклидовых пространств.}

\subsection{Метрические и нормированные пространства}
\begin{definition}
    Множество $M$ называется метрическим пространством, если на нём введена функция $\rho(x, y)$, называемая метрикой,
    которая удовлетворяет следующим свойствам:
    ~\begin{enumerate}
        \item $\forall x \in M: \rho(x, x) = 0$
        \item $\forall x, y \in M: \rho(x, y) = \rho(y, x) \geq 0$
        \item $\forall x, y, z \in M: \rho(x, z) \leq \rho(x, y) + \rho(y, z)$
    \end{enumerate}
\end{definition}

\begin{example}
    В $\mathbb{R}^n$ метрика $\rho(x, y) = \sqrt{\sum_{i = 1}^n (x_i - y_i)^2}$. В пространстве непрерывных функций на
    $[a, b]$ можно взять метрику $\rho(f, g) = \max_{x \in [a, b]} \|f(x) - g(x)\|$
\end{example}

\begin{definition}
    Множество $L$ с операцией сложения и умножения на элемент из $\mathbb{R}$ называется линейным пространством, если
    $\forall x, y, z \in L: \forall \alpha, \beta, \lambda \in \mathbb{R}$: 
    \begin{enumerate}
        \item $\lambda x \in L$
        \item $x + y \in L$
        \item $x + y = y + x$
        \item $x + (y + z) = (x + y) + z$
        \item $\exists 0 \in L : \forall u \in L: u + 0 = 0 + u = u$
        \item $\exists (-x) \in L : x + (-x) = 0$
        \item $\alpha(\beta x) = (\alpha \beta)x$
        \item $\alpha(x + y) = \alpha x + \alpha y$
        \item $1 \cdot x = x$
        \item $(\alpha + \beta)x = \alpha x + \beta x$
    \end{enumerate}
\end{definition}

\begin{definition}
    Линейное пространство $X$ называется нормированным, если на нём определена функция $\| \cdot \|$, такая, что
    $\forall x, y \in X, \lambda \in \mathbb{R}$
    \begin{enumerate}
        \item $\|x\| = 0 \Leftrightarrow x = 0$
        \item $\|\lambda x\| = |\lambda| \|x\|$
        \item $\|x + y\| \leq \|x\| + \|y\|$
    \end{enumerate}
\end{definition}

\begin{remark}
    Нормированное пространство является метрическим, $\rho(x, y) = \|x - y\|$
\end{remark}

\begin{remark}
    Вместо $\mathbb{R}$ можно написать $\mathbb{C}$ или вообще произвольное поле $F$.
\end{remark}


\begin{definition}
    $\varepsilon$ окрестностью точки $x_0$ из нормированного пространства $X$ называется
    \[
        U_{\varepsilon}(x_0) = \{x \in X : \|x - x_0\| < \varepsilon\}
    \]
\end{definition}

\begin{definition}
    Последовательность $\{x_n\}_{n = 1}^{\infty}$ из нормированного пространства $X$ сходится к точке $x_0 \in X$, если
    $\lim_{n \rightarrow \infty} \|x_n - x_0\| = 0$
\end{definition}

\begin{remark}
    Поскольку у нас теперь есть предел, то все определения из $\mathbb{R}^n$ переносятся сюда, а именно: открытые
    множества, замкнутые множества, граница, и т.д.
\end{remark}

\begin{example}
    ~\begin{enumerate}
        \item $C([a, b])$ -- пространство непрерывных на $[a, b]$ функций с нормой $\|f\| = \max_{x \in [a, b]} |f(x)|$ 
        \item $CL_{1} ([a, b])$ -- пространство непрерывных на $[a, b]$ функций с нормной $\|f\| = \int_{a}^{b} |f(x)| dx$
        \item $CL_{2} ([a, b])$ -- пространство непрерывных на $[a, b]$ функций с нормой $\|f\| = \sqrt{\int_{a}^{b}
            |f(x)|^2 dx}$

        \item $RL_{p} ([a, b])$ -- пространство интегрируеммых по Риману на $[a, b]$ функций, с нормой из $CL_{p}$
    \end{enumerate}
\end{example}

\begin{remark}
    Проблема в том, что в $RL_{p}$ есть не тождественно равные нулю функции, у которых интеграл от модуля равен нулю.
    Например, тождественно равная нулю функция, измененная в конечном числе точек. Есть два варианта решения проблемы.
\end{remark}
Первый:
\begin{definition}
    Пространство называется полунормированным, если в нём выполнены все свойства нормированного пространства, кроме
    первого.
\end{definition}
Второй:
\begin{remark}
    Профакторизуем $RL_{p}$ по отношению $f \sim g \Leftrightarrow \left( \int_{a}^{b} (f(x) - g(x))^p dx \right)^{1/p} = 0$. $\hat {RL_{p}} =
    RL_{p}/\sim$
\end{remark}

\subsection{Банаховы пространства}

\begin{definition}
    Пусть $X$ -- метрическое пространство. Последовательность $\{x_n\}_{n = 1}^{\infty}$ называется фундаментальной,
    если $\forall \varepsilon > 0: \exists N \in \mathbb{N}: \forall m, n \geq N: \rho(x_n, x_m) < \varepsilon$
\end{definition}

\begin{definition}
    Метрическое пространство называется полным, если в нём любая фундаментальная последовательность является
    сходящейся.
\end{definition}

\begin{definition}
    Нормированное пространство называется Банаховым, если оно полно по метрике, порожденной нормой.
\end{definition}

\begin{example}
    $C([a, b])$ -- Банахово. Используем теорему Коши для равномерной сходимости функциональных последовательностей
    , а также тот факт, что ряд непрерывных функций, если сходится равномерно, то обязательно к непрерывной функции.
\end{example}

\begin{example}
    $CL_{1} ([a, b]), CL_{2} ([a, b])$ -- не Банаховы. Возьмем последовательность функций 
    \[
        f_n = 
        \begin{cases}
            1, -1 \leq x \leq 0 \\
            1 - xn, 0 \leq x \leq 1/n \\
            0, x > 1/n
        \end{cases}
    \]
    Очевидно, что это последовательность является фундаментальной, ведь $\|f_n - f_m\| \leq \frac{1}{\min \{n, m\}}$.
    Пусть эта последовательнсоть сходится в $CL_1$ к $\varphi(x)$. Тогда $\int_{-1}^{1}|f_n(x) - \varphi(x)|dx \geq
    \int_{-1}^{0}|f_n(x) - \varphi(x)|dx$. Но левый интеграл стемится к нулю, а правый не зависит от $n$. Значит, правый
    интеграл равен нулю. А так как $\varphi$ должна быть непрерывна, $\varphi \equiv 1$ на $[-1, 1]$. Аналогично,
    $\forall 0 < \delta < 1: \int_{-1}^{1}|f_n(x) - \varphi(x)|dx \geq \int_{\delta}^{1}|f_n(x) - \varphi(x)|dx$.
    Аналогично получаем, что $\varphi \equiv 0$ на $[\delta, 1]$. Но это верно для любого $\delta$. Получаем, что
    $\varphi$ разрывна.
\end{example}

\subsection{Евклидовы пространства}

\begin{definition}
    Линейное пространство $R$ называется Евклидовым, если на нём определено скалярное произведение $(\cdot, \cdot)$,
    удовлетворяющее следующим свойствам:
    \begin{enumerate}
        \item $(x, x) \geq 0$
        \item $(x, x) = 0 \Leftrightarrow x = 0$
        \item $(x, y) = (y, x)$
        \item $(\lambda x, y) = \lambda (x, y)$
        \item $(x + y, z) = (x, z) + (y, z)$
    \end{enumerate}
\end{definition}

\begin{remark}
    В Евклидовом пространстве можно ввести норму как $\|x\| = \sqrt{(x, x)}$. Все свойства, кроме неравенства
    треугольника, очевидны. Для неравенства треугольника докажем КБШ.
\end{remark}

\subsection{Неравенство Коши-Буняковского и треугольника для Евклидовых пространств}

\begin{theorem}[Неравенство Коши-Буняковского]
    $(x, y) \leq \sqrt{(x, x) (y, y)}$
\end{theorem}

\begin{proof}
    \[
        0 \leq (x + ty, x + ty) = (x, x) + 2t(x, y) + t^2(y, y)
    \]
    При фиксированных $x, y$ справа написано квадратное уравнение. Значит дискриминант должен быть неположительный.
    $\frac{D}{4} = (x, y)^2 - (x, x)(y, y) \leq 0$
\end{proof}

\begin{corollary}[Неравенство треугольника]
    $(x + y, x + y) = (x, x) + 2(x, y) + (y, y) \leq (x, x) + 2\sqrt{(x, x)(y, y)} + (y, y) = (\sqrt{(x, x)} +
    \sqrt{(y, y)})^2$
\end{corollary}

\begin{example}
    Рассмотрим пространство $CL_{2}$. В нем можно ввести скалярное произведение как $(f, g) = \int_{a}^{b}f(x)g(x)dx$.
    Оно действительно является скалярным произведением, это легко проверить. Тогда норма $\|f\| = \left(
    \int_{a}^{b}|f(x)|^2dx \right)^{1/2}$ действительно является нормой
\end{example}

\end{document}

