\documentclass[document.tex]{subfiles}

\begin{document}
nograhol@gmail.com

Курс состоит из 3 частей:
\begin{enumerate}
\item Теория делимости. Обобщение ОТА (основная теорема арифметики).
\item Расширения полей. Основная теорема алгебры. Конечные поля. Коды БЧХ.
\item Как из $\mathbb{Q}$ перейти в $\mathbb{R}$. $\mathbb{Q}_p$.
\end{enumerate}

\section{Базовые определения.}

\begin{definition}
Кольцо - это тройка $(K, +, \cdot)$. Причем:
\begin{enumerate}
\item $(K, +)$ -- абелева группа
\item $\forall a, b, c \in K: (a + b) \cdot c = a \cdot c + b \cdot c$
\item $\forall a, b, c \in K: c \cdot (a + b) = c \cdot a + c \cdot b$
\end{enumerate}
\end{definition}

\begin{definition}
Свойство $\forall a, b, c: (ab)c = a(bc)$ называют ассоциативностью.
\end{definition}

\begin{definition}
Свойство $\exists 1 : \forall a : a \cdot 1 = 1 \cdot a = a$ называют существованием нейтрального элемента
\end{definition}

\begin{definition}
Свойство $\forall a, b: ab = ba$ называют коммутативностью.
\end{definition}

\begin{definition}
Свойство $\forall a \neq 0 : \exists b : ab = ba = 1$ называют существованием обратного элемента.
\end{definition}

\begin{definition}
Ассоциативное кольцо -- это такое кольцо, что для умножения выполнена ассоциативность.
\end{definition}

\begin{definition}
Кольцо с единицей -- это такое кольцо, где есть нейтральный элемент относительно умножения.
\end{definition}

\begin{definition}
Коммутативное кольцо -- это такое кольцо, что для умножения выполнена коммутативность и (внезапно) ассоциативность и существование нейтрального элемента.
\end{definition}

\begin{remark}
Буквой $K$ будем обозначать коммутативное кольцо (т.е. коммутативное с единицей и ассоциативностью).
\end{remark}

\begin{definition}
Кольцо с обратными -- это такое кольцо, что умножения обратимо.
\end{definition}

\begin{example}
~\begin{enumerate}
\item $\mathbb{Z}$ является коммутативным кольцом с единицей и ассоциативностью
\item $\{0\}$ -- тривиальное кольцо
\item $2\mathbb{Z}$ -- кольцо без единицы, но ассоциативное и коммутативное.
\item $\mathbb{R}^{n \times n}$ -- ассоциативная кольцо с единицей, но не коммутативное
\end{enumerate}
\begin{example}
Более интересный пример:
Множество матриц со сложением и операцией $[\cdot, \cdot]$:
$[A, B] = AB - BA$.
Ассоциативность не выполнена. Но выполнено:
\begin{enumerate}
\item $[[A, B], C] + [[B, C], A] + [[C, A], B] = 0$
\item $[A, B] = -[B, A]$
\end{enumerate}
\end{example}
\end{example}

\begin{definition}
Пусть $K$ -- коммутативное кольцо. Тогда $a \neq 0$ называется делителем нуля, если:
$\exists b \neq 0 : ab = 0$.
\end{definition}

\begin{definition}
Коммутативное кольцо без делителей нуля называется областью целостности.
\end{definition}

\begin{exercise}
$a \cdot 0 = 0$
\end{exercise}

\begin{definition}
$F$ -- поле, если:
\begin{enumerate}
\item $F$ -- ассоциативное коммутативное кольцо с единицей
\item $1 \neq 0$
\item Любой элемент обратим относительно сложения.
\end{enumerate}
\end{definition}

\begin{statement}
В поле нет делителей нуля.
\end{statement}
\begin{proof}
Пусть $a$ -- делитель нуля, т.е. $\exists b \neq 0 : ab = 0$. Но у $a$ есть обратный элемент относительно умножения $a^{-1}$. Умножив слева на $a^{-1}$, придем к противоречию.
\end{proof}

\begin{definition}
Гауссовы числа ($\mathbb{Z}[i]$) -- это комплексные числа с целой мнимой и действительной частью.
\end{definition}

\begin{statement}
Гауссовы числа -- это область целостности
\end{statement}
\begin{proof}
Замкнутость относительно операций проверяется тривиальным образом. Коммутативность, дистрибутивность и ассоциативность следует из соответствующих свойств для $\mathbb{C}$. $0 + 0i$ -- нейтральный элемент относительно сложения, а $1 + 0i$ -- нейтральный элемент относительно умножения, проверяется тривиальным образом. А делителей нуля в гауссовых числах нет, потому что их нет в комплексных числах ($\mathbb{C}$ -- это поле).
\end{proof}

\begin{definition}
Говорят, что $a|b$ ($a$ делит $b$), если $\exists c : ac = b$.
\end{definition}

\begin{statement}
Свойства делимости:
\begin{enumerate}
\item $a|b, b|c \Leftarrow a|c$
\item $a|b, a|c \Leftarrow a|(b+c)$
\item $a|1 \Leftrightarrow \exists b: ab = 1 \Leftrightarrow a \text{ -- обратимый элемент }$
\end{enumerate}
\end{statement}

\begin{remark}
В случае, когда $a|1$, любой элемент поля делится на $a$:

$x = 1 \cdot x = a \cdot a^{-1} \cdot x$
\end{remark}

\begin{definition}
$K^*$ (множество обратимых элементов K) -- мультипликативная группа кольца.
\end{definition}

\begin{definition}
Будем называть два элемента $a$ и $b$ ассоциированными, если $a = rb, r \in K^*$.
\end{definition}

\begin{exercise}
Ассоциированность -- это отношение эквивалентности.
\end{exercise}

\begin{remark}
План доказательства ОТА:
\begin{enumerate}
\item Докажем, что любое число раскладывается на произведение простых.
\item Докажем лемму Евклида.
\item Докажем единственность разложения на простые с помощью леммы Евклида.
\end{enumerate}
\end{remark}

\begin{definition}
Элемент $x \neq 0$ кольца $K$ называется неприводимым или неразложимым, если:
\begin{enumerate}
\item $x \not \in K^*$
\item $x = ab \Rightarrow \exists a^{-1} \vee \exists b^{-1}$
\end{enumerate}
\end{definition}

\begin{definition}
Элемент $0 \neq x \not \in K^*$ кольца $K$ называется простым, если:
$x|ab \Rightarrow x|a \vee x|b$
\end{definition}

\end{document}
