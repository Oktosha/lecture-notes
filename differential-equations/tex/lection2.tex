\documentclass[document.tex]{subfiles}

\begin{document}
\section*{Лекция 2. Интегрирование некоторых ОДУ первого порядка}
\begin{Definition}
Пусть $n \in \mathbb{N}, k \in \mathbb{Z}$. Функция $F : \mathbb{R}^n \mapsto \mathbb{R}$ называется k-однородной (или однородной функцией степени k), если: $\forall \lambda \in \mathbb{R} \setminus \{0\}, v \in \mathbb{R}^n: F(\lambda v) = \lambda^k F(v)$
\end{Definition}
\begin{Definition}
\label{homogen_diff_eq_def1}
Однородным дифференциальным уравнением называется дифференциальное уравнение следующего вида:
$$y' = f(\frac{y}{x})$$
\end{Definition}
\begin{Definition}
\label{homogen_diff_eq_def2}
Однородным дифференциальным уравнение называется дифференциальное уравнение следующего вида:
$$P(x, y)dx + Q(x, y)dy = 0$$,
где $P$ и $Q$ -- однородные функции одного и того же порядка.
\end{Definition}
\begin{Statement}
Определения \ref{homogen_diff_eq_def1} и \ref{homogen_diff_eq_def2} эквивалентны.
\end{Statement}
\end{document}