\documentclass[document.tex]{subfiles}

\begin{document}
\section{Интегрирующие множители и уравнения, не разрешенные относительно производной.}
\begin{definition}
Функция $\mu(x,y)$, определенная в области $G$, называется интегрирующим множителем для уравнения $P(x,y)dx+Q(x,y)dy=0$, если:
\begin{enumerate}
\item $\mu(x,y)\neq0$ в $G$
\item $ \exists U(x,y): \ dU=\mu Pdx+\mu Qdy$
\end{enumerate}
\end{definition}
Частный случай:

Если $P(x,y)$ и $Q(x,y)$ однородные функции степени $n\neq -1$, то 

$\mu(x,y)=\frac{1}{xP(x,y)+yQ(x,y)}$
\subsection{ОДУ первого порядка, не разрешенные относительно производной.}
\begin{theorem}
Если в некотором параллелепипеде в $\mathbb{R}^3$, содержащем точку $(x_0,y_0,y'_0)$, где $y'_0$ -- действительное решение уравнения $F(x_0,y_0,y')=0$, выполнены  следующие условия:
\begin{enumerate}
\item $F(x,y,y')$ определена и непрерывна по совокупности переменных  вместе с производными $\frac{\partial F}{\partial y}$ и $\frac{\partial F}{\partial y'}$
\item $\frac{\partial F}{\partial y'}|_{(x_0,y_0,y'_0)} \neq0$
\end{enumerate}
Тогда в некоторой окрестности $x_0\ \exists!$ решение $y=y(x)$ уравнения $F(x,y,y')=0$ такое, что $y(x_0)=y_0$ и $y'(x_0)=y'_0$.
\end{theorem}
\begin{proof}

Согласно теореме о неявной функции $\exists!$ 
функция $y'=f(x,y)$, удовлетворяющая уравнению $F(x,y,y')=0$, такая,что 
$\frac{\partial f}{\partial y} = - \frac{\frac{\partial F}{\partial y}}{\frac{\partial F}{\partial y'}}$
Тогда по аналогичной теореме для уравнений, разрешенных относительно производной, получаем требуемое.
\end{proof}
\subsection{Метод введения параметра}
Пусть есть уравнение $F(x,y,y')=0$. Тогда:

$F(x,y,y')=0 \Leftrightarrow \left\{\begin{aligned} F(x,y,p)=0\\dy=pdx \end{aligned}\right.$

Пусть $x=\varphi(t),y=\psi(t)$ -- решение $F(x,y,y')=0$. Тогда $p=p(t)=\frac{\psi'(t)}{\varphi'(t)}=y'_x\Rightarrow dy=p(t)dx$, а также $F(x,y,p)\equiv 0$, что и требовалось.

В обратную сторону, если $x=\varphi(t),y=\psi(t)$ -- решение системы, то из второго $p=\frac{dy}{dx}=\frac{\psi'(t)}{\varphi'(t)}\Rightarrow F(x,y,p)\equiv 0$, что и требовалось.
\begin{example}
Рассмотрим уравнения, разрешенные относительно y:$y=f(x,y')$. Тогда $y-f(x,y')=F(x,y,y')=0$
$
\left\{\begin{aligned}dy=pdx\\y=f(x,p) \end{aligned}\right.$

Продифференцируем исходное уравнение по х:
$\frac{dy}{dx}=\frac{\partial f}{\partial x} + \frac{\partial f}{\partial p}\frac{\partial p}{\partial x} = p(x)$

Получили линейное дифференциальное уравнение относительно $p(x)$. Решаем его, получаем $p(x)=\chi(x,c)$. Теперь подставляем это в исходное уравнение и решаем.
\end{example}
\begin{definition}
Множество точек, являющихся решениями уравнения $\frac{\partial F}{\partial p}=0$, называется дискриминантной кривой уравнения.
\end{definition}
\end{document}