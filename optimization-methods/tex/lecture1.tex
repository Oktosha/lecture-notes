\documentclass[document.tex]{subfiles}

\begin{document}
Оценка за зачет:
\begin{enumerate}
\item 40\% от оценки -- за 2 контрольные работы (не переписываются)
\item 30\% от оценки -- за 2 домашние задания
\item 30\% от оценки -- индивидуальный проект, например:
\begin{enumerate}
\item Теоретический (реферат)
\item Теоретико-программистский (анализ времени работы, скорости сходимости)
\item Практический (нужно самому найти данные для применения)
\end{enumerate}
\end{enumerate}
\section{Вводная лекция}
\subsection{Базовые определения}
\begin{definition}[общая задача оптимизации]
Пусть $f: X \mapsto \mathbb{R}$. Нужно найти точку экстремума, т.е. минимума или максимума (локального или глобального) (строго или нестрогого).
\end{definition}
\begin{definition}[задача условной оптимизации]
Пусть $f: Y \mapsto \mathbb{R}$, $X \subset Y$. Нужно минимизировать $f$ на $X$.
\end{definition}
\begin{remark}
Часто $X$ задается условиями вида:
$$\begin{cases}
g_1(x) \leq 0, \\
g_2(x) \leq 0, \\
\ldots \\
g_k(x) \leq 0, \\
g_{k+1}(x) = 0, \\
\ldots \\
g_n(x) = 0.
\end{cases}$$
\end{remark}
\begin{remark}
Методы оптимизации можно условно разделить на аналитические и численные. Например, градиентный спуск -- численный метод, метод Лагранжа -- аналитический. Широкий класс численных методов -- это итеративные алгоритмы. Можно условно разделить итеративные методы на точные и приближённые.
\end{remark}
\subsection{Линейное программирование}
\begin{Definition}
Задача линейного программирования -- минимизация линейной функции на многограннике.

Более строго: пусть дана линейная функция $f: \mathbb{R}^n \mapsto \mathbb{R}$, имеющая вид $$f(x_1, \ldots, x_n) = \sum_{i=1}^n \alpha_i x_i$$. Пусть также дана система линейных уравнений и неравенств: $A_1x \leq b_1$, $A_2x = b_2$. Задача стоит в нахождении минимума $f$ на множестве, на котором выполнена система уравнений.
\end{Definition}
\begin{definition}
Систему линейных уравнений и неравенств $A_1x \leq b_1$, $A_2x \leq b_2$ назовём системой ограничений.
\end{definition}
\begin{definition}
Ограничения со знаком неравенства будем называть уравнениями-неравенствами.
\end{definition}
\begin{definition}
Ограничения со знаком равенства будем называть уравнениями-равенствами.
\end{definition}
\begin{Example}
Производственная задача: даны товары $g_1, \ldots, g_n$ и ресурсы $r_1, \ldots r_m$. Ресурсов ограниченное число. Ресурсов $i$-того типа: $\omega_i$. На производство $g_i$ необходимо $c_{i, j}$ ресурсов $r_j$. $p_i$ -- цена $g_i$. Нужно максимизировать прибыль.

Обозначим $x_i$ -- сколько товаров $g_i$ было произведено. Тогда есть следующая задача максимизации:
$$\max \sum_{i=0}^n p_i x_i$$
$$\begin{cases}
x_1 \geq 0 \\
\ldots \\
x_n \geq 0 \\
\sum_{j=1}^n x_j c_{j,1} \leq \omega_1 \\
\ldots \\
\sum_{j=1}^n x_j c_{j,m} \leq \omega_m
\end{cases}$$
\end{Example}

\begin{remark}
Сначала считаем, что уравнений-равенств нет.
\end{remark}

\begin{definition}
Грань $k$-той размерности -- это множество точек, в которой ровно $n-k$ неравенств обратились в равенство, а остальные неравенства верны.
\end{definition}

\begin{statement}
Если минимум достигается, то он достигается на какой-то грани.
\end{statement}

\begin{statement}
Если минимум достигается во внутренней точке грани, то он достигается на всей грани.
\end{statement}

\begin{corollary}
Если минимум достигается, то есть вершина многогранника, в которой он достигается.
\end{corollary}

\begin{corollary}
Есть экспоненциальный алгоритм решения задачи линейного программирования -- простой перебор всех вершин.
\end{corollary}

\begin{remark}
Есть симплекс метод, который работает за $m^3$.
\end{remark}


\end{document}