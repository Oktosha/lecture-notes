\documentclass{article}

%\usepackage{subfiles}
\usepackage{../../mystyle}

\begin{document}
%\customtitle{Теория колец и полей}{Андрей Степанов}{Ильинский Д.}
%\newpage
%\tableofcontents
%\newpage
%\subfile{notation}
%\newpage
%\subfile{lecture-1}
%\subfile{lecture-2}
%\subfile{lecture-3}

\section{Определения}

\begin{definition}
    Кольцо -- это тройка $(R, +, *)$, где $R$ -- непустое множество, $+, *: R^2 \mapsto R$, такая что $(R, +)$ --
    абелева группа, а также выполнена дистрибутивность умножения $*$ относительно сложения $+$ слева и справа.
    Нейтральный элемент относительно сложения обозначается $0$

    Кольцо с единицей -- это кольцо, в котором относительно умножения есть нейтральный элемент, обозначаемый $1$: $1*a =
    a*1 = 1$

    Ассоциативное кольцо -- это кольцо, в котором выполнена ассоциативность операции умножения: $a*(b*c) = (a*b)*c$

    Коммутативное кольцо -- это кольцо, в котором выполнена коммутативность операции умножения $a*b = b*a$, а также
    присутствует единица и выполнена ассоциативность.
\end{definition}

\begin{definition}
    Элемент $a \neq 0$ ассоциативного кольца с единицей $R$ называется обратимым, если $\exists a^{-1} \in  R: a^{-1} *
    a = a * a^{-1} = 1$
\end{definition}

\begin{definition}
    Элемент $0 \neq a \in R$ называется делителем нуля, если $\exists 0 \neq b \in R: ab = 0$
\end{definition}

\begin{definition}
    Для кольца $K$ множество его обратимых элементов обозначается $K^*$

    Элементы $a$ и $b$ называются ассоциированными, если $\exists c \in K^*: a = cb$
\end{definition}

\begin{definition}
    Коммутативное кольцо без делителей нуля называется областью целостности.
\end{definition}

\begin{definition}
    Ненулевой необратимый элемент $a$ области целостности называется неразложимым, если из того, что он представляется в
    виде $a = bc$, следует, что либо $b$ либо $c$ обратим.
\end{definition}

\begin{definition}
    Ненулевой необратимый элемент $p$ называется простым, если из того, что $p | ab$ следует, что либо $p | a$ либо $p |
    b$
\end{definition}

\begin{definition}
    Евклидово кольцо -- это область целостности $K$ с определенной на ней функцией евклидовой нормы $N: K \setminus
    \{0\} \mapsto \mathbb{N}_0$:
    \begin{enumerate}
        \item $\forall a, b \in K \setminus \{0\}: N(a) \leq N(ab)$
        \item $\forall a, b \in K \setminus \{0\}: \exists q, r: a = qb + r, N(r) < N(b)$
    \end{enumerate}
\end{definition}

\begin{definition}
    Пусть $K$ -- область целостности. Тогда элемент $z \in K$ называется наибольшим общим делителем элементов $a, b \in
    K$ (обозначается как $(a, b)$), если $z|a, z|b$ и $\forall z': z'|a, z'|b$ выполнено, что  $z'|z$
\end{definition}

\begin{definition}
    Пусть $R_1$ и $R_2$ -- кольца. Отображение $\varphi: R_1 \mapsto R_2$ наызвается гомоморфизмом колец, если:
    \begin{enumerate}
        \item $\varphi(a+b) = \varphi(a) + \varphi(b)$
        \item $\varphi(a*b) = \varphi(a) * \varphi(b)$
    \end{enumerate}
\end{definition}

\begin{definition}
    Подмножество $R \subset K$ называется подкольцом, если оно замкнуто относительно умножения и является подгруппой по
    сложению.
\end{definition}

\begin{definition}
    Подкольцо $R$ коммутативного кольца $K$ называется идеалом, если оно замкнуто относительно умножения на элемент из
    $K$, то есть $\forall r \in R, k \in K: rk \in R$
\end{definition}

\begin{definition}
    Тривиальным называют идеал, либо совпадающий со всем кольцом, либо состоящий из одного элемента (нейтрального
    элемента по сложению)
\end{definition}

\begin{definition}
    Идеал $I$ коммутативного кольца $K$ называется порожденным элементами $x_1, \cdots, x_n$ (обозначение $I = (x_1,
    \cdots, x_n)$), если $I = \{a_1 * x_1 + a_2 * x_2 + \cdots + a_n * x_n | \forall i: a_i \in K\}$
\end{definition}

\begin{definition}
    Идеал конечнопорожден, если он порожден конечным числом элементов.
\end{definition}

\begin{definition}
    Идеал называется главным, если он порожден одним элементом.
\end{definition}

\begin{definition}
    Кольцо называется кольцом главных идеалов (КГИ), если в нём все идеалы главные.
\end{definition}

\begin{definition}
    Область целостности называется факториальным кольцом, если в нём любой ненулевой элемент либо обратим, либо с
    точностью до перестановки и домножения на обратимые представляется в виде произведения неразложимых.
\end{definition}

\begin{definition}
    Идеал $I \neq K$ называется простым, если $ab \in I \Rightarrow a \in I \vee b \in I$
\end{definition}

\begin{definition}
    Идеал $I \neq K$ называется максимальным, если не существует другого нетривиального идеала, содержащего $I$
\end{definition}

\section{Вопросы сложности 2}

\begin{statement}
    В коммутативном кольце элемент не может иметь двух различных обратных
\end{statement}

\begin{proof}
    Пусть $K$ -- коммутативное кольцо, $a \in K$ -- ненулевой элемент этого кольца, $a_1, a_2$ -- два различных обратных
    элемента к нему.
    Тогда, с одной стороны $a_1 a a_2 = a_1 (a a_2) = a_1$, а с другой стороны $a_1 a a_2 = (a_1 a) a_2 = a_2$.
    Получили, что $a_1 = a_2$. Противоречие.
\end{proof}

\begin{statement}
    Пусть $R$ -- кольцо с единицей, причем $|R| > 1$. Тогда в этом кольце $1 \neq 0$
\end{statement}

\begin{proof}
    Пусть $a \in R$. Докажем, что $a \cdot 0 = 0$. Воспользуемся тем, что $0 = 0 + 0$ (это прямое следствие аксиом
    кольца), а также
    дистрибутивностью:
    \[
        a \cdot 0 = a \cdot (0 + 0) = a \cdot 0 + a \cdot 0
    \]
    Если добавить к обоим частям равнества обратный по сложению $-(a \cdot 0)$, то получим, что $a \cdot 0 = 0$.

    Пусть теперь $1 = 0$. Поскольку $|R| > 1$, то можно найти такой $a \in R$, что $a \neq 0$. Тогда $a \cdot 1 = 0$ из
    выше доказанного. С другой стороны, поскольку $1$ -- нейтральный элемент по умножению, $a \cdot 1 = a$. Тогда $a =
    0$. Но мы выбирали $a$ так, что $a \neq 0$. Противоречие.
\end{proof}

\begin{statement}
    Пусть $R$ -- ассоциативное кольцо с единицей. $a$ -- обратимый элемент в $R$. Тогда $a$ не может быть делителем нуля.
\end{statement}

\begin{proof}
    Пусть $\exists b \neq 0: ab = 0$. Умножим последнее равенство на $a^{-1}$. Тогда $0 = a^{-1} ab = (a^{-1}a)b = b$.
    Получили, что $b = 0$. Противоречие.
\end{proof}

\begin{statement}
    Пусть $K$ -- область целостности, пусть $a, b, c \in K$, причем $c \neq 0$. Тогда $ac = bc \Rightarrow a = b$
\end{statement}

\begin{proof}
    $ac = bc \Leftrightarrow ac - bc = 0 \Leftrightarrow (a - b)c = 0$. Поскольку $K$ -- область целостности, то либо $c
    = 0$, либо $a - b = 0$. Но первое противоречит условию, поэтому верно второе, то есть $a = b$.
\end{proof}

\begin{statement}
    $S = \{\frac{p}{q} \in \mathbb{Q}: (p, 1) = 1, q | n\}$ не является подкольцом $\mathbb{Q}$
\end{statement}

\begin{proof}
    Пусть $n = 12$, $\frac{1}{4} \in S, \frac{1}{6} \in S$. Но их произведение $\frac{1}{4} \cdot \frac{1}{6} =
    \frac{1}{24} \not \in S$. Получили, что $S$ не замкнуто относительно умножения.
\end{proof}

\begin{statement}
    Пусть $p$ -- простое, $S = \{\frac{a}{b} \in \mathbb{Q}: (a, b) = 1, \not p | b\}$. Тогда $S$ -- подкольцо в
    $\mathbb{Q}$
\end{statement}

\begin{proof}
    Проверяем замкнутость относительно операций. Пусть $\frac{a}{b} \in S, \frac{c}{d} \in S$, причем $\not p | b, \not
    p | d$
    $\frac{a}{b} + \frac{c}{d} = \frac{ad + bc}{bd}$, причем $\not p | bd$. Действительно, пусть $p | bd$. Так как $p$
    простое, то либо $p | b$, либо $p | d$. А это не так. Поскольку $\not p | bd$, то и после сокращения дроби
    $\frac{ad + bc}{bd}$ на некоторое число $e$, $\not p | \frac{bd}{e}$. Действительно, воспользуемся ОТА: пусть $bd =
    p_1 p_2 \cdots p_k$, причем в этом разложении нет числа $p$. Но тогда после сокращения, в разложении числа $bd$
    могут лишь исчезнуть некоторые $p_i$, но не появится $p$.

    Аналогично с произведением. Понятно также, что все обратные к $\frac{a}{b}$ в $S$ лежат, ведь это просто
    $\frac{-a}{b}$
\end{proof}

\begin{statement}
    Пусть $p$ -- простое, $S = \{\frac{a}{b} \in \mathbb{Q}: (a, b) = 1, \exists n \in \mathbb{N}_0: b = p^n\}$. Тогда
    $S$ -- подкольцо в $\mathbb{Q}$
\end{statement}

\begin{proof}
    Проверяем замкнутость операций, пусть $\frac{a}{p^n} \in S, \frac{b}{p^m} \in S$. Без ограничения общности, $m > n$.

    Тогда $\frac{a}{p^n} + \frac{b}{p^m} = \frac{ap^{m - n} + b}{p^m}$. После сокращения последней дроби её знаменатель
    останется степенью $p$. Аналогично с произведением. Обратные ко всем элементам также лежат.
\end{proof}

\begin{statement}
    Множество обратимых элементов ассоциативного кольца с единицей является группой по умножению и называется
    мультипликативной группой кольца
\end{statement}

\begin{proof}
    Пусть $K^*$ -- это множество всех обратимых элементов ассоциативного кольца с единицей $K$. Понятно, что для этих
    элементов выполняется ассоциативность, ведь она наследуется из кольца $K$. Кроме того, $1 \in K^*$, ведь $1$ --
    обратимый элемент. И последнее: если $a$ -- обратим, то $a^{-1}$ тоже обратим. В итоге мы доказали, что $K^*$ -- группа.
\end{proof}

\begin{statement}
    $a \sim b \Leftrightarrow a | b \wedge b | a$
\end{statement}

\begin{proof}
    Пусть $a \sim b$. Тогда $\exists c \in K^*: a = bc$. Тогда $b | a$. Кроме того, $c^{-1}a = b$, то есть $a | b$.

    Наоборот, пусть $a | b, b | a$. Понятно, что тогда $a \neq 0, b \neq 0$. Тогда $b = ca, a = db$. Тогда $b = cdb$.
    Сокращая на $b$ получаем, что $cd = 1$, а это означает, что $c$ и $d$ -- обратимые, то есть $a \sim b$
\end{proof}

\begin{statement}
    Если $a$ -- неразложим, а $b \sim a$, то $b$ -- неразложим.
\end{statement}

\begin{proof}
    Пусть $b$ -- разложимый элемент, то есть $\exists c, d \not \in K^*: b = cd$. Но $a = eb$, причем $e \in K^*$. Тогда
    $a = ecd$. Но $ec \not \in K^*$. Действительно, пусть $ec \in K^*$. $e^{-1} \in K^*$. Тогда $c \in K^*$, а это не
    так. Получили разложения для $a$ на необратимые элементы.
\end{proof}

\begin{statement}
    Пусть $p$ -- простой, $p \sim q$. Тогда $q$ тоже простой.
\end{statement}

\begin{proof}
    Пусть $q | ab, q = cp, c \in K^*$. Тогда $ab = dq = cdp$. Тогда $p | ab$. Тогда либо $p | a$, либо $p | b$. Пусть,
    без ограничения общности, $p | a$. Тогда $a = ep = e c^{-1}q$. Но тогда $q | a$.
\end{proof}

\begin{statement}
    Пусть $d_1 = (a, b), d_2 = (a, b)$. Тогда $d_1 \sim d_2$
\end{statement}

\begin{proof}
    Поскольку $d_1$ -- наибольший общий делитель, а $d_2$ -- общий делитель, то $d_2 | d_1$. Аналогично, $d_1 | d_2$. По
    критерию ассоциированности, $d_1 \sim d_2$
\end{proof}

\begin{statement}
    $\mathbb{Z}[\omega]$ -- евклидово кольцо с нормой $N(a + b \omega) = a^2 + b^2 - ab$
\end{statement}

\begin{proof}
    Заметим, что $|a + b \omega|^2 = (a + b \omega) \cdot (a + b \overline \omega) = a^2 + ab (\omega + \overline
    \omega) + b^2 \omega \overline \omega = a^2 + b^2 - ab = N(a + b \omega) \geq 1$, при $(a, b) \neq 0$

    Тогда для $z_1, z_2 \neq 0$: $N(z_1 z_2) = z_1 z_2 \overline z_1 \overline z_2 = N(z_1) N(z_2) \geq N(z_1)$ и первое
    свойство нормы выполнено.

    Теперь нужно сказать пару слов, про то, как мы делим элементы в $\mathbb{Z}[\omega]$ (то есть как для любых двух $a,
    b \in \mathbb{Z}[\omega]$
    выбрать $q, r \in \mathbb{Z}[\omega]$ так, что $a = bq + r$, причем $N(r) < N(b)$)

    Положим $q = \left[\frac{a}{b}\right]$ -- ближайшую к $\frac{a}{b}$ точку из $\mathbb{Z}[\omega]$, $r = b * \left(q
    - \frac{a}{b}\right) = b * \left(\left[ \frac{a}{b} \right] - \frac{a}{b}\right) = bq - a$. Если мы докажем, что
    $\left| \left[ \frac{a}{b} \right] - \frac{a}{b} \right| < 1$, это будет означать, что $N(r) < N(1) N(b) = N(b)$.
    Для этого докажем, что расстояние вообще от любой точки из $\mathbb{C}$ до ближайшей точки $\mathbb{Z}[\omega]$
    удовлетворяет требуемому неравенству. Рассмотрим $z \in \mathbb{C}$. Для неё в $\mathbb{Z}[\omega]$ есть три ближайшие
    точки $z_1, z_2, z_3$, образующие треугольник вокруг $z$. Любой такой треугольник является равносторонним со стороной 1.
    Докажем, что $f(z) = \max_{z} \min \{|z_1 - z|, |z_2 - z|, |z_3 - z|\} < 1$. Но максимум достигается, когда все $|z_i - z|$
    равны. Тогда точка $z$ -- центр описанной окружности вокрут треугольника, а $f(z)$ -- то радиус описанной
    окружности, который находится по формуле $\frac{abc}{4S} = \frac{1}{\sqrt{3}} < 1$
\end{proof}

\begin{statement}
    \label{div_by_natural}
    В области целостности $\mathbb{Z}[u]$ элемент $z = a + bu$ делится на $k \in \mathbb{Z}$ тодгда и только тогда,
    когда $a$ и $b$ делятся на $k$.
\end{statement}

\begin{proof}
    Пусть $k|z$. Тогда $a + bu = z = k(x + yu) = kx + kyu$. Пусть $u = c + di$. Тогда $a + bc + bdi = kx + kyc + kydi$.
    Два компексных числа равны, если равны их мнимые и действительные части, поэтому
    \[
        \begin{cases}
            a + bc = kx + kyc, \\
            bd = kyd
        \end{cases}
    \]
    Считаем, что $d \neq 0$, в противном случае утверждение не верно (например, $2 | 1 + 3 = 4$, но неверно, что $2|1,
    2|3$).
    Тогда $b = ky, a = kx$. Значит, $a$ и $b$ делятся на $k$.

    Пусть наоборот, $a$ и $b$ делятся на $k$. Тогда $b = ky, a = kx$. $z = a + bu = k(x + uy)$. Тогда $k|z$.
\end{proof}

\begin{statement}
    В $\mathbb{Z}[\omega]$ если $z | x, |z| = |x|$, то $z \sim x$
\end{statement}

\begin{proof}
    $x = zy$, причем $|x| = |z| |y|$, а значит, $|y| = 1$. Но в $\mathbb{Z}[\omega]$ все такие $z$, что $|z| = 1$
    обратимы, следовательно, $z \sim x$
\end{proof}

\begin{statement}
    В $\mathbb{Z}[i]$ если $z | x, |z| = |x|$, то $z \sim x$
\end{statement}

\begin{proof}
    $x = zy$, причем $|x| = |z| |y|$, а значит, $|y| = 1$. Но в $\mathbb{Z}[i]$ все такие $z$, что $|z| = 1$
    обратимы, следовательно, $z \sim x$
\end{proof}

\begin{statement}
    Если $z$ -- неразложимый в $\mathbb{Z}[i]$, то $\exists p$ -- простое, $N(z) = p \vee N(z) = p^2$
\end{statement}

\begin{proof}
    Будет пользоваться тем фактом, что $\mathbb{Z}[i]]$ -- факториальное кольцо. Тогда $z$ -- простой. $N(z) = z
    \overline z$, причем $N(z) \in \mathbb{Z}$. Разложим $N(z)$ на простые. $z \overline z = p_1^{k_1} p_2^{k_2} \cdots
    p_s^{k_s}$. То есть $z | p_1^{k_1} \cdots p_s^{k_s}$. но $z$ -- простое, поэтому $\exists i: z | p_i$. То есть $z x
    = p_i$. Обозначим $p = p_i$, оно простое. $N(z) N(x) = p^2$. Есть 3 варианта:
    \begin{enumerate}
        \item $N(x) = 1$. Тогда $N(z) = p^2$
        \item $N(x) = p$. Тогда $N(z) = p$.
        \item $N(x) = p^2$. Тогда $N(z) = 1$, и $z$ обратим, а значит, не является неразложимым. Противоречие.
    \end{enumerate}
\end{proof}

\begin{statement}
    Если $z$ -- неразложимый в $\mathbb{Z}[\omega]$, то $\exists p$ -- простое, $N(z) = p \vee N(z) = p^2$
\end{statement}

\begin{proof}
    Доказательство повторяет предыдущее.
    Будет пользоваться тем фактом, что $\mathbb{Z}[\omega]$ -- факториальное кольцо. Тогда $z$ -- простой. $N(z) = z
    \overline z$, причем $N(z) \in \mathbb{Z}$. Разложим $N(z)$ на простые. $z \overline z = p_1^{k_1} p_2^{k_2} \cdots
    p_s^{k_s}$. То есть $z | p_1^{k_1} \cdots p_s^{k_s}$. но $z$ -- простое, поэтому $\exists i: z | p_i$. То есть $z x
    = p_i$. Обозначим $p = p_i$, оно простое. $N(z) N(x) = p^2$. Есть 3 варианта:
    \begin{enumerate}
        \item $N(x) = 1$. Тогда $N(z) = p^2$
        \item $N(x) = p$. Тогда $N(z) = p$.
        \item $N(x) = p^2$. Тогда $N(z) = 1$, и $z$ обратим, а значит, не является неразложимым. Противоречие.
    \end{enumerate}
\end{proof}

\begin{statement}
    Если $x$ -- неразложимый элемент $\mathbb{Z}[i]$ и $N(z) = p^2$, то $z \sim p$
\end{statement}

\begin{proof}
    В рамках предыдущего доказательства мы показали, что $\exists x: zx = p$. Тогда $N(z) N(x) = p^2$, но также $N(z) =
    p^2$, а значит, $N(x) = 1$. Значит, $x$ -- обратим и $z \sim p$
\end{proof}


\begin{statement}
    Если $x$ -- неразложимый элемент $\mathbb{Z}[\omega]$ и $N(z) = p^2$, то $z \sim p$
\end{statement}

\begin{proof}
    Аналогично.
\end{proof}

\begin{statement}
    Если для $z \in \mathbb{Z}[i]$ выполнено, что $N(z) = p$, где $p$ -- простое, то $z$ неразложим.
\end{statement}

\begin{proof}
    Пусть $z$ разложим, тогда $z = z_1 z_2$, причем $z_1, z_2 \not \in \mathbb{Z}[i]^*$. Тогда $p = N(z) = N(z_1)
    N(z_2)$. Так как $p$ простое, то либо $N(z_1) = 1$, либо $N(z_2) = 1$. Но тогда либо $z_1$, либо $z_2$ обратим.
    Противоречие.
\end{proof}

\begin{statement}
    Если для $z \in \mathbb{Z}[\omega]$ выполнено, что $N(z) = p$, где $p$ -- простое, то $z$ неразложим.
\end{statement}

\begin{proof}
    Аналогично.
\end{proof}

\begin{statement}
    Множество делителей нуля кольца $K$ вместе с нулём не всегда образуют идеал.
\end{statement}

\begin{proof}
    Рассмотрим $K = \mathbb{Z}_6$. Его множество делителей нуля (вместе с нулем) -- это $\{0, 2, 3\}$. Это множество не
    образует даже подкольцо, так как $2 + 3 = 5 \not \in \{0, 2, 3\}$
\end{proof}

\begin{statement}
    3 -- разложимый элемент $\mathbb{Z}[\omega]$
\end{statement}

\begin{proof}
    $(1 - \omega)(1 - \omega^2) = 1 - \omega - \omega^2 + \omega^3 = 2 - \omega - \omega^2 = 2 - \omega - (-1 -
    \omega) = 3$
\end{proof}

\begin{statement}
    Если идеал $I \subset K$ содержит обратимый элемент, то $I = K$
\end{statement}

\begin{proof}
    Пусть $a \in I$ -- обратимый элемент. Тогда $\exists a^{-1} \in K: a a^{-1} = 1$. Из опеределения идеала $\forall x
    \in I: \forall y \in K: xy \in I$. Значит, $1 = a a^{-1} \in I$. Раз $1 \in I$, то и $\forall y \in K: 1 \cdot y \in
    I$. Значит, $K \subset I$. Но тогда $K = i$.
\end{proof}

\begin{statement}
    $I = (a_1, \cdots, a_k) = \{x_1 a_1 + \cdots + x_k a_k : \forall i : x_i \in K \}$ -- это минимальный по включению
    идеал, содержащий элементы $a_1, \cdots, a_k$.
\end{statement}

\begin{proof}
    Во-первых, $I$ -- это идеал. Действительно, пусть $x \in I, y \in K$. Тогда $x = x_1 a_1 + \cdots x_k a_k$. $yx = y
    x_1 a_1 + \cdots + y x_k a_k \in I$. Кроме того, это подгруппа по сложению.

    Пусть $J$ -- другой идеал, содержащий $a_1, \cdots, a_k$. Тогда $\forall i: \forall x \in K: x a_i \in J$. Тогда
    $\forall x_1, \cdots, x_k: x_1 a_1 + \cdots + x_k a_k \in J$. Но тогда $I \subset J$. Но это и означает, что $I$ --
    минмальный по включению идеал, содержащий элементы $a_1, \cdots, a_k$.
\end{proof}

\begin{statement}
    Идеал $(x, x + 1) \subset \mathbb{Z}[x]$ не является ни простым, ни максимальным.
\end{statement}

\begin{proof}
    $x \in (x, x+1), x+1 \in (x, x+1) \Rightarrow x+1 - x = 1 \in (x, x+1)$. Но тогда $I = \mathbb{Z}[x]$. То есть этот
    идеал тривиальный. Значит, он не максимальный и не простой.
\end{proof}

\section{Вопросы сложности 3}

\begin{statement}
    Множество $S = \{x + \sqrt{2}y : x, y \in \mathbb{Q}\}$ является кольцом.
\end{statement}

\begin{proof}
    Так как $S \subset \mathbb{R}$, а $\mathbb{R}$ -- кольцо, то достаточно проверить замкнутость $S$.
    Пусть $x + \sqrt{2}y \in S, a + \sqrt{2}b \in S$. Тогда $-(x + \sqrt{2}y) = -x + \sqrt{2}(-y) \in S$. $(x +
    \sqrt{2}y) + (a + \sqrt{2}b) = (x + a) + \sqrt{2}(y + b) \in S$. $(x + \sqrt{2}y)(a + \sqrt{2}b) = (xa + 2yb) +
    \sqrt{2}(ya + xb) \in S$. Получаем, что $S$ замкнуто относительно операци. Значит $S$ -- подкольцо, значит $S$ --
    кольцо.
\end{proof}

\begin{statement}
    Простой элемент области целостности является неразложимым.
\end{statement}

\begin{proof}
    Пусть $p$ -- простой элемент области целостности $K$. Пусть $p$ -- разложим, то есть $\exists a \not in K^*, b \not
    \in K^*: p = ab$. Тогда $p | ab$. Значит, либо $p | a$, либо $p | b$. Пусть без ограничения общности $p | a$. Тогда
    $a = px$. Тогда $p = pxb$. Значит, $p(1 - xb) = 0$. Так как $p \neq 0$, то $1 - xb = 0$. Значит, $xb = 1$, и
    следовательно, $b \in K^*$. Противоречие.
\end{proof}

\begin{statement}
    При каких $u \in \mathbb{C}$ множество $\mathbb{Z}[u] = \{a + bu : a, b \in \mathbb{Z}\}$ является областью
    целостности.
\end{statement}

\begin{proof}
    Заметим, что $\mathbb{Z}[u] \subset \mathbb{C}$. Но в $\mathbb{C}$ делителей нуля нет, так как это поле (ну или так:
    пусть $a$ -- делители нуля в $\mathbb{C}$, тогда $0 = |ab| = |a||b|$. Но тогда либо $|a| = 0$, либо $|b| = 0$).

    Осталось проверить, при каких $u$ $\mathbb{Z}[u]$ замкнуто. Понятно, что $(a + bu) + (c + du) = (a + c) + (d + b)u$,
    то есть относительно сложения это множество всегда замкнуто. Посмотрим, что происходит при умножении:
    $(a + bu)(c + du) = ac + (bc + ad)u + bd u^2$. Значит, это множество замкнуто тогда и только тогда, когда $u^2 \in
    \mathbb{Z}[u]$. То есть $\exists r, s: u^2 = r + su$. Заметим, что если $u$ -- корень $u^2 = r + su$, то и
    $\overline u$ это тоже корень $u^2 = r + su$. Тогда по теореме Виета это означает, что $u + \overline u = 2 \Re u \in
    \mathbb{Z}, u \cdot \overline u = |u|^2 \in \mathbb{Z}$
\end{proof}

\begin{statement}
    \[
        \mathbb{Z}[ni]^* = 
        \begin{cases}
            \{1, -1, i, -i\}, n = 1, \\
            \{1, -1\}, n > 1. \\
        \end{cases}
    \]
\end{statement}

\begin{proof}
    Заметим, что если $z$ -- обратимый, то $|z|^2 |z^{-1}|^2 = |zz^{-1}|^2 = |1|^2 = 1$. Если $z, z^{-1} \in \mathbb{Z}[ni]$,
    то $|z|^2, |z^{-1}|^2 \in \mathbb{Z}$. Произведение двух положительных чисел из $\mathbb{Z}$ дает единицу, если оба
    числа это единица. Значит, обратимыми могут быть только элементы с нормой $1$. Просто переберём все элементы из
    $\mathbb{Z}[ni]$ с нормой $1$ и посмотрим, какие из них обратимы.
\end{proof}

\begin{statement}
    $\mathbb{Z}[\omega]^* = \{1, -1, \omega, -\omega, \omega^2, -\omega^2\}$
\end{statement}

\begin{proof}
    Рисуем $\mathbb{Z}[\omega]$ на листочке и внимательно смотрим, использую соображения из предыдущего доказательства.
\end{proof}


\begin{statement}
    $\mathbb{Z}[3i]$ не факториально.
\end{statement}

\begin{proof}
    $3i \cdot (-3i) = 9 = 3 \cdot 3$. Докажем, что $3, 3i, -3i$ неразложимы. Пусть не так, и скажем, $3 = z_1 z_2$,
    причем ни $z_1$, ни $z_2$ не обратимы, а следовательно $N(z_1) > 1, N(z_2) > 1$.
    Тогда $9 = N(z_1) N(z_2)$. Понятно, что тогда $N(z_1) = 3, N(z_2) = 3$. Но $N(z) = a^2 + 9b^2$. Легко показать, что
    $N(z) \neq 3$ при любых $a, b$. Ну или можно нарисовать $\mathbb{Z}[3i]$ и убедиться в этом при помощи геометрии.
    Оба варианта являются правильными. Аналогично делаем с $3i$ и $-3i$
\end{proof}

\begin{statement}
    $\mathbb{Z}[\sqrt{3}i]$ не факториально.
\end{statement}

\begin{proof}
    $4 = 2 \cdot 2 = (1 - \sqrt{3}i)(1 + \sqrt{3}i)$. $N(2) = N(1 - \sqrt{3}i) = N(1 + \sqrt{3}i) = 4$. Покажем, что
    элемент с нормой 4 неразложим. $4 = N(z_1) \cdot N(z_2)$. Тогда $N(z_1) = N(z_2) = 2$. Но элементов с такой нормой в
    $\mathbb{Z}[\sqrt{3}i]$ нет.
\end{proof}

\begin{statement}
    Если $N(ab) = N(a)$, и $a, b \neq 0$, то $b$ -- обратим
\end{statement}

\begin{proof}
    Разделим $a$ на $ab$ с остатком. Тогда $\exists q, r: a = abq + r$. Пусть $r \neq 0$. Тогда $N(r) < N(ab) = N(a)$.
    Но $r = a - abq = a (1 - bq)$, следовательно $a | r$. Тогда $r = xa$. $N(r) = N(xa) \geq N(a)$. Но мы получили
    противоречие, ведь $N(a) \leq N(xa) = N(r) < N(ab) = N(a)$. Значит $r = 0$. Но тогда $b$ -- обратимый
\end{proof}

\begin{statement}
    Если $b$ -- обратим, то $N(ab) = N(a)$
\end{statement}

\begin{proof}
    Из свойства нормы: $N(ab) \geq N(a)$. Докажем, что $N(a) \geq N(ab)$. Так как $b$ -- обратимый, то $a = a b
    b^{-1}$. Тогда Из свойства нормы $N(a) = N(abb^{-1}) \geq N(ab)$. Конец.
\end{proof}

\begin{statement}
    Если $p$ -- простое целое число, причем $p = 4k + 3$, то $p$ -- неразложимый элемент в $\mathbb{Z}[i]$
\end{statement}

\begin{proof}
    Пусть не так. Тогда $\exists z_1, z_2 \not \in \mathbb{Z}[i]^*: p = z_1 z_2$. Посмотрим на норму $p$: $p^2 = N(z_1)
    N(z_2)$. Понятно, что без ограничения общности есть два варианта:
    \begin{enumerate}
        \item $N(z_1) = N(z_2) = p$. Но такого быть не может, т.к. $N(z_1) = a^2 + b^2 \not \equiv 3 \mod 4$
        \item $N(z_1) = 1, N(z_2) = p^2$. Но тогда $z_1$ обратимый, и мы опять пришли к противоречию.
    \end{enumerate}
    Значит, $p$ неразложимый.
\end{proof}

\begin{statement}
    Если $p$ -- простое целое число вида $4k + 1$, то $p$ -- разложимый элемент в $\mathbb{Z}[i]$
\end{statement}

\begin{proof}
    Предположим противное, пусть $p$ -- неразложимый, а следовательно, простой, так как $\mathbb{Z}[i]$ -- факториально.
    
    Заметим, что $-1$ является квадратичным вычетом по модулю $p$ (другими словами, $\exists a: a^2 \equiv -1 \mod p$).
    Это можно понять, посчитав символ Лежандра $\left(\frac{-1}{p}\right)$. Но есть и другой вариант доказательства. Мы
    знаем, что $\forall a: a^{p - 1} \equiv 1 \mod p$ из малой теоремы Ферма. Тогда $(a^{\frac{p - 1}{2}} - 1)
    (a^{\frac{p - 1}{2}} + 1) \equiv 0 \mod p$. Мы знаем, что у этого многочлена $p - 1$ корень, а у
    $a^{\frac{p - 1}{2}} - 1 \equiv 0 \mod p$ не может быть больше $\frac{p - 1}{2}$ корней (так как $\mathbb{Z}[p]$ --
    это поле). Тогда $\exists a: a^{\frac{p - 1}{2}} \equiv -1 \mod p$. Если положить $x = a^{k}$, то $x^2 =
    a^{2k} = a^{\frac{p - 1}{2}} \equiv -1 \mod p$. Значит, $-1$ является квадратичным вычетом по модулю $p$. Тогда $x^2
    + 1 = (x - i)(x + i) \equiv 0 \mod p$. Так как $p$ простое, то либо $p | x + i$, либо $p | x - i$. Но оба этих
    утверждения неверны (это доказывалось в \ref{div_by_natural})
\end{proof}

\begin{statement}
    Натуральное число представимо в виде суммы двух квадратов (целых чисел) тогда и только тогда, когда
    любое простое число вида $4k + 3$ входит в его разложение на простые множители в чётной степени.
\end{statement}

\begin{proof}
    Пусть число $m$ представимо в виде суммы двух квадратов, тогда $\exists z: m = z \overline z$. Разложим $z$ на
    простые. Пусть $z = p_1 \cdots p_s$. Тогда $m = (p_1 \overline p_1) \cdots (p_s \overline p_s)$. Почему в этом
    разложении простые вида $4k+3$ входят в чётных степени? Давайте это проверим, путь для некоторого $k$ простое $4k+3$
    входит в разложение $m$ на простые,
    тогда $4k+3|m$. Так как $4k+3$ простое над $\mathbb{Z}[i]$, то либо $\exists i: 4k+3|p_i$, либо $\exists i: 4k+3|\overline p_i$.
    Пусть без
    ограничения общности $4k+3|p_i$, то есть $(4k+3)x = p_i$. Но $p_i$ -- простое (и $\overline p_i$). Тогда они
    неразложимы, а тогда $x$ -- обратимо. Но тогда $p_i \sim 4k+3$. Но $N(p_i) = p_i \overline p_i = (4k+3)^2$. Значит,
    эта скобка $(p_i \overline p_i) = (4k+3)^2$. Поделим $m$ на $(4k+3)^2$ и продолжим доказательство по индукции. В
    результате, все простые вида $(4k+3)$, которые нам удастся вынести, будут всегда выносится в чётной степени. 

    Пусть теперь наоборот, $m$ таково, что простые вида $4k+3$ входят в его разложение в чётной степени. То есть $m =
    p_1^2 \cdots p_s^2 q_1 \cdots q_r$. Причем $p1, \cdots, p_s$ -- простые вида $4k+3$, а $q_1, \cdots, q_r$ -- простые
    вида $4k+1$. Докажем, что $\forall i \in \{1, \cdots, r\}: \exists z_i: q_i = z_i \cdot \overline z_i$.
    Дейстивтельно, $q_i$ -- это простое вида $4k+1$. Оно разложимо над $\mathbb{Z}[i]$. Тогда $q_i = uv$, тогда $q_i^2 =
    N(q_i) = N(u)N(v)$. Но $N(u) > 1, N(v) > 1$. Тогда $u \overline u = N(u) = q_i, v \overline v = N(v) = q_i$. Тогда
    можно переписать разложение для $m$ в виде: $m = p_1^2 \cdots p_s^2 (z_1 \overline z_1) \cdots (z_r \overline z_r)$.
    Если положить $z = p_1 \cdots p_s z_1 \cdots z_r$, то $m = z \overline z = N(z)$, а значит $m$ представляется как сумма
    квадратов.
\end{proof}

\begin{statement}
    Пусть $p$ -- простое целое число вида $p = 3k+1$. Тогда $p$ разложим в $\mathbb{Z}[\omega]$
\end{statement}

\begin{proof}
    Сначала докажем, что $-3$ является квадратичным вычетом по модулю $p$ используя символ Лежандра:
    $\left(\frac{-3}{p}\right) = \left(\frac{-1}{p}\right) \cdot \left( \frac{3}{p} \right) =
    (-1)^{\frac{p - 1}{2}} \left( \frac{p}{3} \right) (-1)^{\frac{p - 1}{2}} = \left( \frac{p}{3} \right) = \left(
    \frac{1}{3} \right) = 1$. Здесь мы воспользовались квадратичным законом взаимности и критерием Эйлера для
    квадратичных вычетов. Значит, $\exists c: c^2 + 3 = (c - \sqrt{3}i)(c + \sqrt{3}i) \equiv 0 \mod p$. Но
    $\sqrt{3}i = 2 \omega + 1$. Тогда $p | (c + 1 + 2 \omega)(c - 1 - 2\omega)$. Если бы $p$ было простым, то либо $p|(c
    + 1 + 2\omega)$, либо $p|(c - 1 - 2\omega)$. Но это не так (показано в \ref{div_by_natural})
\end{proof}

\begin{statement}
    Если $p$ -- простое число вида $p = 3K+2$, то $p$ -- неразложимый элемент $\mathbb{Z}[\omega]$
\end{statement}

\begin{proof}
    Пусть не так и $p = z_1 z_2$, причем $N(z_1) > 1, N(z_2) > 1$. Тогда $p^2 = N(z_1) N(z_2)$. Это возможно, если
    только если $N(z_1) = N(z_2) = p$. Узнаем, можно ли найти такие $a, b: a^2 - ab + b^2 = p = 3k+2$. Посмотрим на это
    равенство по модулю $3$. Тогда $2 \equiv a^2 - ab + b^2 \equiv a^2 + 2ab + b^2 \equiv (a + b)^2 \mod 3$. Но $2$ --
    не квадратичный вычет по модулю $3$. Значит, таких $a, b$ найти не удастся, значит, $p$ -- неразложимый. 
\end{proof}

\end{document}
