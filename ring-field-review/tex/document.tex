\documentclass{article}

%\usepackage{subfiles}
\usepackage{../../mystyle}

\begin{document}
%\customtitle{Теория колец и полей}{Андрей Степанов}{Ильинский Д.}
%\newpage
%\tableofcontents
%\newpage
%\subfile{notation}
%\newpage
%\subfile{lecture-1}
%\subfile{lecture-2}
%\subfile{lecture-3}

\section{Определения}

\begin{definition}
    Кольцо -- это тройка $(R, +, *)$, где $R$ -- непустое множество, $+, *: R^2 \mapsto R$, такая что $(R, +)$ --
    абелева группа, а также выполнена дистрибутивность умножения $*$ относительно сложения $+$ слева и справа.
    Нейтральный элемент относительно сложения обозначается $0$

    Кольцо с единицей -- это кольцо, в котором относительно умножения есть нейтральный элемент, обозначаемый $1$: $1*a =
    a*1 = 1$

    Ассоциативное кольцо -- это кольцо, в котором выполнена ассоциативность операции умножения: $a*(b*c) = (a*b)*c$

    Коммутативное кольцо -- это кольцо, в котором выполнена коммутативность операции умножения $a*b = b*a$, а также
    присутствует единица и выполнена ассоциативность.
\end{definition}

\begin{definition}
    Элемент $a \neq 0$ ассоциативного кольца с единицей $R$ называется обратимым, если $\exists a^{-1} \in  R: a^{-1} *
    a = a * a^{-1} = 1$
\end{definition}

\begin{definition}
    Элемент $0 \neq a \in R$ называется делителем нуля, если $\exists 0 \neq b \in R: ab = 0$
\end{definition}

\begin{definition}
    Для кольца $K$ множество его обратимых элементов обозначается $K^*$

    Элементы $a$ и $b$ называются ассоциированными, если $\exists c \in K^*: a = cb$
\end{definition}

\begin{definition}
    Коммутативное кольцо без делителей нуля называется областью целостности.
\end{definition}

\begin{definition}
    Ненулевой необратимый элемент $a$ области целостности называется неразложимым, если из того, что он представляется в
    виде $a = bc$, следует, что либо $b$ либо $c$ обратим.
\end{definition}

\begin{definition}
    Ненулевой необратимый элемент $p$ называется простым, если из того, что $p | ab$ следует, что либо $p | a$ либо $p |
    b$
\end{definition}

\begin{definition}
    Евклидово кольцо -- это область целостности $K$ с определенной на ней функцией евклидовой нормы $N: K \setminus
    \{0\} \mapsto \mathbb{N}_0$:
    \begin{enumerate}
        \item $\forall a, b \in K \setminus \{0\}: N(a) \leq N(ab)$
        \item $\forall a, b \in K \setminus \{0\}: \exists q, r: a = qb + r, N(r) < N(b)$
    \end{enumerate}
\end{definition}

\begin{definition}
    Пусть $K$ -- область целостности. Тогда элемент $z \in K$ называется наибольшим общим делителем элементов $a, b \in
    K$ (обозначается как $(a, b)$), если $z|a, z|b$ и $\forall z': z'|a, z'|b$ выполнено, что  $z'|z$
\end{definition}

\begin{definition}
    Пусть $R_1$ и $R_2$ -- кольца. Отображение $\varphi: R_1 \mapsto R_2$ наызвается гомоморфизмом колец, если:
    \begin{enumerate}
        \item $\varphi(a+b) = \varphi(a) + \varphi(b)$
        \item $\varphi(a*b) = \varphi(a) * \varphi(b)$
    \end{enumerate}
\end{definition}

\begin{definition}
    Подмножество $R \subset K$ называется подкольцом, если оно замкнуто относительно умножения и является подгруппой по
    сложению.
\end{definition}

\begin{definition}
    Подкольцо $R$ коммутативного кольца $K$ называется идеалом, если оно замкнуто относительно умножения на элемент из
    $K$, то есть $\forall r \in R, k \in K: rk \in R$
\end{definition}

\begin{definition}
    Тривиальным называют идеал, либо совпадающий со всем кольцом, либо состоящий из одного элемента (нейтрального
    элемента по сложению)
\end{definition}

\begin{definition}
    Идеал $I$ коммутативного кольца $K$ называется порожденным элементами $x_1, \cdots, x_n$ (обозначение $I = (x_1,
    \cdots, x_n)$), если $I = \{a_1 * x_1 + a_2 * x_2 + \cdots + a_n * x_n | \forall i: a_i \in K\}$
\end{definition}

\begin{definition}
    Идеал конечнопорожден, если он порожден конечным числом элементов.
\end{definition}

\begin{definition}
    Идеал называется главным, если он порожден одним элементом.
\end{definition}

\begin{definition}
    Кольцо называется кольцом главных идеалов (КГИ), если в нём все идеалы главные.
\end{definition}

\begin{definition}
    Область целостности называется факториальным кольцом, если в нём любой ненулевой элемент либо обратим, либо с
    точностью до перестановки и домножения на обратимые представляется в виде произведения неразложимых.
\end{definition}

\begin{definition}
    Нетривиальный идеал $I$ называется простым, если $ab \in I \Rightarrow a \in I \vee b \in I$
\end{definition}

\begin{definition}
    Нетривиальный идеал $I$ называется максимальным, если не существует другого нетривиального идеала, содержащего $I$
\end{definition}

\section{Вопросы сложности 2}

\begin{statement}
    В коммутативном кольце элемент не может иметь двух различных обратных
\end{statement}

\begin{proof}
    Пусть $K$ -- коммутативное кольцо, $a \in K$ -- ненулевой элемент этого кольца, $a_1, a_2$ -- два различных обратных
    элемента к нему.
    Тогда, с одной стороны $a_1 a a_2 = a_1 (a a_2) = a_1$, а с другой стороны $a_1 a a_2 = (a_1 a) a_2 = a_2$. 
    Получили, что $a_1 = a_2$. Противоречие.
\end{proof}

\begin{statement}
    Пусть $R$ -- кольцо с единицей, причем $|R| > 1$. Тогда в этом кольце $1 \neq 0$
\end{statement}

\begin{proof}
    Пусть $a \in R$. Докажем, что $a \cdot 0 = 0$. Воспользуемся тем, что $0 = 0 + 0$ (это прямое следствие аксиом
    кольца), а также
    дистрибутивностью:
    \[
        a \cdot 0 = a \cdot (0 + 0) = a \cdot 0 + a \cdot 0
    \]
    Если добавить к обоим частям равнества обратный по сложению $-(a \cdot 0)$, то получим, что $a \cdot 0 = 0$.

    Пусть теперь $1 = 0$. Поскольку $|R| > 1$, то можно найти такой $a \in R$, что $a \neq 0$. Тогда $a \cdot 1 = 0$ из
    выше доказанного. С другой стороны, поскольку $1$ -- нейтральный элемент по умножению, $a \cdot 1 = a$. Тогда $a =
    0$. Но мы выбирали $a$ так, что $a \neq 0$. Противоречие.
\end{proof}

\begin{statement}
    Пусть $R$ -- ассоциативное кольцо с единицей. $a$ -- обратимый элемент в $R$. Тогда $a$ не может быть делителем нуля.
\end{statement}

\begin{proof}
    Пусть $\exists b \neq 0: ab = 0$. Умножим последнее равенство на $a^{-1}$. Тогда $0 = a^{-1} ab = (a^{-1}a)b = b$.
    Получили, что $b = 0$. Противоречие.
\end{proof}

\begin{statement}
    Пусть $K$ -- область целостности, пусть $a, b, c \in K$, причем $c \neq 0$. Тогда $ac = bc \Rightarrow a = b$
\end{statement}

\begin{proof}
    $ac = bc \Leftrightarrow ac - bc = 0 \Leftrightarrow (a - b)c = 0$. Поскольку $K$ -- область целостности, то либо $c
    = 0$, либо $a - b = 0$. Но первое противоречит условию, поэтому верно второе, то есть $a = b$.
\end{proof}

\begin{statement}
    $S = \{\frac{p}{q} \in \mathbb{Q}: (p, 1) = 1, q | n\}$ не является подкольцом $\mathbb{Q}$
\end{statement}

\begin{proof}
    Пусть $n = 12$, $\frac{1}{4} \in S, \frac{1}{6} \in S$. Но их произведение $\frac{1}{4} \cdot \frac{1}{6} =
    \frac{1}{24} \not \in S$. Получили, что $S$ не замкнуто относительно умножения.
\end{proof}

\begin{statement}
    Пусть $p$ -- простое, $S = \{\frac{a}{b} \in \mathbb{Q}: (a, b) = 1, \not p | b\}$. Тогда $S$ -- подкольцо в
    $\mathbb{Q}$
\end{statement}

\begin{proof}
    Проверяем замкнутость относительно операций. Пусть $\frac{a}{b} \in S, \frac{c}{d} \in S$, причем $\not p | b, \not
    p | d$
    $\frac{a}{b} + \frac{c}{d} = \frac{ad + bc}{bd}$, причем $\not p | bd$. Действительно, пусть $p | bd$. Так как $p$
    простое, то либо $p | b$, либо $p | d$. А это не так. Поскольку $\not p | bd$, то и после сокращения дроби
    $\frac{ad + bc}{bd}$ на некоторое число $e$, $\not p | \frac{bd}{e}$. Действительно, воспользуемся ОТА: пусть $bd =
    p_1 p_2 \cdots p_k$, причем в этом разложении нет числа $p$. Но тогда после сокращения, в разложении числа $bd$
    могут лишь исчезнуть некоторые $p_i$, но не появится $p$.

    Аналогично с произведением. Понятно также, что все обратные к $\frac{a}{b}$ в $S$ лежат, ведь это просто
    $\frac{-a}{b}$
\end{proof}

\begin{statement}
    Пусть $p$ -- простое, $S = \{\frac{a}{b} \in \mathbb{Q}: (a, b) = 1, \exists n \in \mathbb{N}_0: b = p^n\}$. Тогда
    $S$ -- подкольцо в $\mathbb{Q}$ 
\end{statement}

\begin{proof}
    Проверяем замкнутость операций, пусть $\frac{a}{p^n} \in S, \frac{b}{p^m} \in S$. Без ограничения общности, $m > n$.

    Тогда $\frac{a}{p^n} + \frac{b}{p^m} = \frac{ap^{m - n} + b}{p^m}$. После сокращения последней дроби её знаменатель
    останется степенью $p$. Аналогично с произведением. Обратные ко всем элементам также лежат.
\end{proof}

\begin{statement}
    Множество обратимых элементов ассоциативного кольца с единицей является группой по умножению и называется
    мультипликативной группой кольца
\end{statement}

\begin{proof}
    Пусть $K^*$ -- это множество всех обратимых элементов ассоциативного кольца с единицей $K$. Понятно, что для этих
    элементов выполняется ассоциативность, ведь она наследуется из кольца $K$. Кроме того, $1 \in K^*$, ведь $1$ --
    обратимый элемент. И последнее: если $a$ -- обратим, то $a^{-1}$ тоже обратим. В итоге мы доказали, что $K^*$ -- группа.
\end{proof}

\begin{statement}
    $a \sim b \Leftrightarrow a | b \wedge b | a$
\end{statement}

\begin{proof}
    Пусть $a \sim b$. Тогда $\exists c \in K^*: a = bc$. Тогда $b | a$. Кроме того, $c^{-1}a = b$, то есть $a | b$.

    Наоборот, пусть $a | b, b | a$. Понятно, что тогда $a \neq 0, b \neq 0$. Тогда $b = ca, a = db$. Тогда $b = cdb$.
    Сокращая на $b$ получаем, что $cd = 1$, а это означает, что $c$ и $d$ -- обратимые, то есть $a \sim b$ 
\end{proof}

\begin{statement}
    Если $a$ -- неразложим, а $b \sim a$, то $b$ -- неразложим.
\end{statement}

\begin{proof}
    Пусть $b$ -- разложимый элемент, то есть $\exists c, d \not \in K^*: b = cd$. Но $a = eb$, причем $e \in K^*$. Тогда
    $a = ecd$. Но $ec \not \in K^*$. Действительно, пусть $ec \in K^*$. $e^{-1} \in K^*$. Тогда $c \in K^*$, а это не
    так. Получили разложения для $a$ на необратимые элементы.
\end{proof}

\begin{statement}
    Пусть $p$ -- простой, $p \sim q$. Тогда $q$ тоже простой.
\end{statement}

\begin{proof}
    Пусть $q | ab, q = cp, c \in K^*$. Тогда $ab = dq = cdp$. Тогда $p | ab$. Тогда либо $p | a$, либо $p | b$. Пусть,
    без ограничения общности, $p | a$. Тогда $a = ep = e c^{-1}q$. Но тогда $q | a$.
\end{proof}

\begin{statement}
    Пусть $d_1 = (a, b), d_2 = (a, b)$. Тогда $d_1 \sim d_2$
\end{statement}

\begin{proof}
    Поскольку $d_1$ -- наибольший общий делитель, а $d_2$ -- общий делитель, то $d_2 | d_1$. Аналогично, $d_1 | d_2$. По
    критерию ассоциированности, $d_1 \sim d_2$
\end{proof}

\begin{statement}
    $\mathbb{Z}[\omega]$ -- евклидово кольцо с нормой $N(a + b \omega) = a^2 + b^2 - ab$
\end{statement}

\begin{proof}
    Заметим, что $|a + b \omega|^2 = (a + b \omega) \cdot (a + b \overline \omega) = a^2 + ab (\omega + \overline
    \omega) + b^2 \omega \overline \omega = a^2 + b^2 - ab = N(a + b \omega)$
\end{proof}

\end{document}
